%% Generated by Sphinx.
\def\sphinxdocclass{report}
\documentclass[letterpaper,12pt,english]{sphinxmanual}
\ifdefined\pdfpxdimen
   \let\sphinxpxdimen\pdfpxdimen\else\newdimen\sphinxpxdimen
\fi \sphinxpxdimen=.75bp\relax
\ifdefined\pdfimageresolution
    \pdfimageresolution= \numexpr \dimexpr1in\relax/\sphinxpxdimen\relax
\fi
%% let collapsible pdf bookmarks panel have high depth per default
\PassOptionsToPackage{bookmarksdepth=5}{hyperref}

\PassOptionsToPackage{booktabs}{sphinx}
\PassOptionsToPackage{colorrows}{sphinx}

\PassOptionsToPackage{warn}{textcomp}
\usepackage[utf8]{inputenc}
\ifdefined\DeclareUnicodeCharacter
% support both utf8 and utf8x syntaxes
  \ifdefined\DeclareUnicodeCharacterAsOptional
    \def\sphinxDUC#1{\DeclareUnicodeCharacter{"#1}}
  \else
    \let\sphinxDUC\DeclareUnicodeCharacter
  \fi
  \sphinxDUC{00A0}{\nobreakspace}
  \sphinxDUC{2500}{\sphinxunichar{2500}}
  \sphinxDUC{2502}{\sphinxunichar{2502}}
  \sphinxDUC{2514}{\sphinxunichar{2514}}
  \sphinxDUC{251C}{\sphinxunichar{251C}}
  \sphinxDUC{2572}{\textbackslash}
\fi
\usepackage{cmap}
\usepackage[T1]{fontenc}
\usepackage{amsmath,amssymb,amstext}
\usepackage{babel}



\usepackage{tgtermes}
\usepackage{tgheros}
\renewcommand{\ttdefault}{txtt}



\usepackage[Bjarne]{fncychap}
\usepackage{sphinx}

\fvset{fontsize=auto}
\usepackage{geometry}

\usepackage{nbsphinx}

% Include hyperref last.
\usepackage{hyperref}
% Fix anchor placement for figures with captions.
\usepackage{hypcap}% it must be loaded after hyperref.
% Set up styles of URL: it should be placed after hyperref.
\urlstyle{same}

\addto\captionsenglish{\renewcommand{\contentsname}{Contents:}}

\usepackage{sphinxmessages}
\setcounter{tocdepth}{1}



\title{Using Jupyter Notebooks, MyST Markdown, \& Sphinx to Publish Professional Quality HTML \& LaTeX Documents}
\date{Jan 08, 2024}
\release{}
\author{N.\@{} L.\@{} Dentinger\sphinxhyphen{}Anderson, C.\@{} L.\@{} Dentinger\sphinxhyphen{}Valentine}
\newcommand{\sphinxlogo}{\vbox{}}
\renewcommand{\releasename}{}
\makeindex
\begin{document}

\ifdefined\shorthandoff
  \ifnum\catcode`\=\string=\active\shorthandoff{=}\fi
  \ifnum\catcode`\"=\active\shorthandoff{"}\fi
\fi

\pagestyle{empty}
\sphinxmaketitle
\pagestyle{plain}
\sphinxtableofcontents
\pagestyle{normal}
\phantomsection\label{\detokenize{index::doc}}


\sphinxstepscope

\begin{sphinxVerbatim}[commandchars=\\\{\}]
\PYG{n+nt}{title}\PYG{p}{:}\PYG{+w}{ }\PYG{l+lScalar+lScalarPlain}{Automating}\PYG{l+lScalar+lScalarPlain}{ }\PYG{l+lScalar+lScalarPlain}{Setup}
\PYG{n+nt}{description}\PYG{p}{:}\PYG{+w}{ }\PYG{l+lScalar+lScalarPlain}{This}\PYG{l+lScalar+lScalarPlain}{ }\PYG{l+lScalar+lScalarPlain}{Notebook}\PYG{l+lScalar+lScalarPlain}{ }\PYG{l+lScalar+lScalarPlain}{will}\PYG{l+lScalar+lScalarPlain}{ }\PYG{l+lScalar+lScalarPlain}{illustrate}\PYG{l+lScalar+lScalarPlain}{ }\PYG{l+lScalar+lScalarPlain}{how}\PYG{l+lScalar+lScalarPlain}{ }\PYG{l+lScalar+lScalarPlain}{we}\PYG{l+lScalar+lScalarPlain}{ }\PYG{l+lScalar+lScalarPlain}{can}\PYG{l+lScalar+lScalarPlain}{ }\PYG{l+lScalar+lScalarPlain}{automate}\PYG{l+lScalar+lScalarPlain}{ }\PYG{l+lScalar+lScalarPlain}{the}\PYG{l+lScalar+lScalarPlain}{ }\PYG{l+lScalar+lScalarPlain}{build}\PYG{l+lScalar+lScalarPlain}{ }\PYG{l+lScalar+lScalarPlain}{of}\PYG{l+lScalar+lScalarPlain}{ }\PYG{l+lScalar+lScalarPlain}{this}\PYG{l+lScalar+lScalarPlain}{ }\PYG{l+lScalar+lScalarPlain}{entire}\PYG{l+lScalar+lScalarPlain}{ }\PYG{l+lScalar+lScalarPlain}{Jupyter}\PYG{l+lScalar+lScalarPlain}{ }\PYG{l+lScalar+lScalarPlain}{setup}\PYG{l+lScalar+lScalarPlain}{ }\PYG{l+lScalar+lScalarPlain}{in}\PYG{l+lScalar+lScalarPlain}{ }\PYG{l+lScalar+lScalarPlain}{a}\PYG{l+lScalar+lScalarPlain}{ }\PYG{l+lScalar+lScalarPlain}{custom}\PYG{l+lScalar+lScalarPlain}{ }\PYG{l+lScalar+lScalarPlain}{conda}\PYG{l+lScalar+lScalarPlain}{ }\PYG{l+lScalar+lScalarPlain}{environment.}
\PYG{n+nt}{date}\PYG{p}{:}\PYG{+w}{ }\PYG{l+lScalar+lScalarPlain}{2024\PYGZhy{}01\PYGZhy{}05}
\end{sphinxVerbatim}

\begin{sphinxuseclass}{cell}
\begin{sphinxuseclass}{cell_input}
\begin{sphinxVerbatim}[commandchars=\\\{\}]
\PYG{o}{\PYGZpc{}\PYGZpc{}html}
\PYG{p}{\PYGZlt{}}\PYG{n+nt}{style}\PYG{p}{\PYGZgt{}}
\PYG{+w}{    }\PYG{n+nt}{body}\PYG{+w}{ }\PYG{p}{\PYGZob{}}
\PYG{+w}{        }\PYG{n+nv}{\PYGZhy{}\PYGZhy{}vscode\PYGZhy{}font\PYGZhy{}family}\PYG{p}{:}\PYG{+w}{ }\PYG{l+s+s2}{\PYGZdq{}lmroman17\PYGZhy{}regular\PYGZdq{}}
\PYG{+w}{    }\PYG{p}{\PYGZcb{}}
\PYG{p}{\PYGZlt{}}\PYG{p}{/}\PYG{n+nt}{style}\PYG{p}{\PYGZgt{}}
\end{sphinxVerbatim}

\end{sphinxuseclass}
\begin{sphinxuseclass}{cell_output}
\begin{sphinxVerbatim}[commandchars=\\\{\}]
\PYGZlt{}IPython.core.display.HTML object\PYGZgt{}
\end{sphinxVerbatim}

\end{sphinxuseclass}
\end{sphinxuseclass}

\chapter{How to Automate Python Environment}
\label{\detokenize{notebooks/01-automating-setup:how-to-automate-python-environment}}\label{\detokenize{notebooks/01-automating-setup::doc}}\begin{quote}

\sphinxAtStartPar
\sphinxstylestrong{Objective}

\sphinxAtStartPar
Create a custom Conda environment and integrate into Jupyter Notebooks with a single command.
\end{quote}


\section{Outline of Project}
\label{\detokenize{notebooks/01-automating-setup:outline-of-project}}
\sphinxAtStartPar
I want to automate the steps that it took to configure the Lua Programming/Jupyter Notebook \sphinxcode{\sphinxupquote{conda}} environment as there were a few annoying bits to the installation. I would like to prevent such hiccups in the future and also learn how to create build scripts for future environments instead of always installing programs and packages to my global/base environment.


\subsection{Prerequisites}
\label{\detokenize{notebooks/01-automating-setup:prerequisites}}\begin{itemize}
\item {} 
\sphinxAtStartPar
User has \sphinxcode{\sphinxupquote{conda}} installed on their system.

\item {} 
\sphinxAtStartPar
User will want to specify the Python version every time they create a new environment.

\end{itemize}


\subsection{Actions of Program}
\label{\detokenize{notebooks/01-automating-setup:actions-of-program}}\begin{itemize}
\item {} 
\sphinxAtStartPar
Create environment of specified name and Python version.

\item {} 
\sphinxAtStartPar
Install \sphinxcode{\sphinxupquote{ipykernel}} to the Python environment.

\item {} 
\sphinxAtStartPar
Link environment to Jupyter kernel.

\end{itemize}

\sphinxAtStartPar
Once all of these actions are complete, we can then activate the environment and install whatever packages we need and immediately use them in Jupyter.


\section{Developing the Script}
\label{\detokenize{notebooks/01-automating-setup:developing-the-script}}
\sphinxAtStartPar
First, in the root directory, create a new folder called \sphinxcode{\sphinxupquote{scripts}}. This will allow us to put any other scripts we make in this folder as well. Let’s create two scripts \sphinxcode{\sphinxupquote{01\sphinxhyphen{}create\sphinxhyphen{}env.sh}} and \sphinxcode{\sphinxupquote{02\sphinxhyphen{}build\sphinxhyphen{}env.sh}}.

\sphinxAtStartPar
Assuming this root directory is the current workspace open in VS Code, your directory structure will look like the following:

\sphinxAtStartPar
\sphinxincludegraphics[width=946\sphinxpxdimen,height=1176\sphinxpxdimen]{{lua-dir}.jpg}

\sphinxAtStartPar
Make sure you know where your \sphinxcode{\sphinxupquote{bash}} is located. I am working in an Apple Silicon architecture and prefer to use \sphinxcode{\sphinxupquote{homebrew}} to manage my packages rather than relying on the native MacOS installations of \sphinxcode{\sphinxupquote{python}}, \sphinxcode{\sphinxupquote{bash}}, etc. To install the latest version of \sphinxcode{\sphinxupquote{bash}} just run:

\begin{sphinxVerbatim}[commandchars=\\\{\}]
brew\PYG{+w}{ }install\PYG{+w}{ }bash
\end{sphinxVerbatim}

\sphinxAtStartPar
Then locate your \sphinxcode{\sphinxupquote{.bashrc}} file and add the \sphinxcode{\sphinxupquote{\$BREW\_PREFIX}} to the \sphinxcode{\sphinxupquote{\$PATH}} variable with the following line:

\begin{sphinxVerbatim}[commandchars=\\\{\}]
\PYG{n+nb}{export}\PYG{+w}{ }\PYG{n+nv}{PATH}\PYG{o}{=}\PYG{l+s+s2}{\PYGZdq{}}\PYG{l+s+s2}{/opt/homebrew/bin:/opt/homebrew/sbin:}\PYG{n+nv}{\PYGZdl{}PATH}\PYG{l+s+s2}{\PYGZdq{}}
\end{sphinxVerbatim}

\sphinxAtStartPar
Now, when you run the \sphinxcode{\sphinxupquote{which bash}} command you should get the following output:

\begin{sphinxVerbatim}[commandchars=\\\{\}]
which\PYG{+w}{ }bash
\end{sphinxVerbatim}

\sphinxAtStartPar
\sphinxcode{\sphinxupquote{/opt/homebrew/bin/bash}}

\sphinxAtStartPar
Using a \sphinxstyleemphasis{shebang} or \sphinxcode{\sphinxupquote{\#!}}, we can put it in front of our path to ensure that our files are executable. This will will be the first line in both of our scripts.

\sphinxAtStartPar
We will use two scripts to get this job done. The first script will just create the virtual environment with its name (first parameter) and the Python version (second parameter).

\sphinxAtStartPar
While this is only one line, we are placing this in a different file because the execution ends after the environment is completed. This is where the second file comes in, allowing us to call the first script to create the environment, then using the rest of the script to build the Jupyter kernel.

\sphinxAtStartPar
Please find the complete scripts here:
\begin{itemize}
\item {} 
\sphinxAtStartPar
\DUrole{xref,download,myst}{Script 1 (\sphinxcode{\sphinxupquote{01\sphinxhyphen{}create\sphinxhyphen{}env.sh}})}

\item {} 
\sphinxAtStartPar
\DUrole{xref,download,myst}{Script 2 (\sphinxcode{\sphinxupquote{02\sphinxhyphen{}build\sphinxhyphen{}env.sh}})}

\end{itemize}

\sphinxAtStartPar
Here is what the first file looks like:

\begin{sphinxVerbatim}[commandchars=\\\{\}]
\PYG{c+ch}{\PYGZsh{}! /opt/homebrew/bin/bash}
conda\PYG{+w}{ }create\PYG{+w}{ }\PYGZhy{}n\PYG{+w}{ }\PYG{n+nv}{\PYGZdl{}1}\PYG{+w}{ }\PYG{n+nv}{python}\PYG{o}{=}\PYG{n+nv}{\PYGZdl{}2}
\end{sphinxVerbatim}

\sphinxAtStartPar
If you are not familiar with \sphinxcode{\sphinxupquote{bash}} syntax, \sphinxcode{\sphinxupquote{\$1}} represents the first parameter that is needed to run the file, whereas \sphinxcode{\sphinxupquote{\$2}} represents the second parameter.

\sphinxAtStartPar
For the second file, we will call \sphinxcode{\sphinxupquote{01\sphinxhyphen{}create\sphinxhyphen{}env.sh}} and continue the kernel creation process. Here’s what that looks like:

\begin{sphinxVerbatim}[commandchars=\\\{\}]
\PYG{c+ch}{\PYGZsh{}! /opt/homebrew/bin/bash}
\PYG{c+c1}{\PYGZsh{} Run this script in the root directory as follows where \PYGZlt{}PROJECT\PYGZus{}PREFIX\PYGZgt{} is the absolute path to root directory or the output of pwd command: }
\PYG{c+c1}{\PYGZsh{} scripts/02\PYGZhy{}build\PYGZhy{}env.sh \PYGZlt{}PROJECT\PYGZus{}PREFIX\PYGZgt{}}

\PYG{c+c1}{\PYGZsh{} Make configuration directory for }
mkdir\PYG{+w}{ }./.config

\PYG{c+c1}{\PYGZsh{} Environment has been created. Now, we install the necessary packages in it. }
bash\PYG{+w}{ }./scripts/01\PYGZhy{}create\PYGZhy{}env.sh\PYG{+w}{ }\PYG{n+nv}{\PYGZdl{}1}

\PYG{c+c1}{\PYGZsh{} Source Miniconda so we can activate the env in the bash script.}
\PYG{n+nb}{source}\PYG{+w}{ }/opt/homebrew/Caskroom/miniconda/base/etc/profile.d/conda.sh

\PYG{c+c1}{\PYGZsh{} Activate env and install ipykernel}
conda\PYG{+w}{ }activate\PYG{+w}{ }jupyter\PYGZhy{}lua
conda\PYG{+w}{ }install\PYG{+w}{ }ipykernel

\PYG{c+c1}{\PYGZsh{} Link environment to Jupyter}
python\PYG{+w}{ }\PYGZhy{}m\PYG{+w}{ }ipykernel\PYG{+w}{ }install\PYG{+w}{ }\PYGZhy{}\PYGZhy{}user\PYG{+w}{ }\PYGZhy{}\PYGZhy{}name\PYG{o}{=}jupyter\PYGZhy{}lua

\PYG{c+c1}{\PYGZsh{} Initialize node.js}
npm\PYG{+w}{ }init

\PYG{c+c1}{\PYGZsh{} Install ijavascript kernel}
npm\PYG{+w}{ }install\PYG{+w}{ }ijavascript
ijinstall

\PYG{c+c1}{\PYGZsh{} Change to the conda\PYGZus{}config directory for next few steps.}
\PYG{n+nb}{cd}\PYG{+w}{ }\PYG{n+nv}{\PYGZdl{}1}/.config

\PYG{c+c1}{\PYGZsh{} Install xeus\PYGZhy{}lua kernel which has a dependency on xcanvas. This dependency must be installed from source.}
git\PYG{+w}{ }clone\PYG{+w}{ }git@github.com:jupyter\PYGZhy{}xeus/xcanvas.git
\PYG{n+nb}{cd}\PYG{+w}{ }xcanvas\PYG{+w}{ }\PYG{o}{\PYGZam{}\PYGZam{}}\PYG{+w}{ }mkdir\PYG{+w}{ }build\PYG{+w}{ }\PYG{o}{\PYGZam{}\PYGZam{}}\PYG{+w}{ }\PYG{n+nb}{cd}\PYG{+w}{ }build
cmake\PYG{+w}{ }\PYGZhy{}D\PYG{+w}{ }\PYG{n+nv}{CMAKE\PYGZus{}INSTALL\PYGZus{}PREFIX}\PYG{o}{=}\PYG{n+nv}{\PYGZdl{}CONDA\PYGZus{}PREFIX}\PYG{+w}{ }..
make\PYG{+w}{ }install

\PYG{c+c1}{\PYGZsh{} Go back to the conda\PYGZus{}config directory}
\PYG{n+nb}{cd}\PYG{+w}{ }\PYG{n+nv}{\PYGZdl{}1}/.config

\PYG{c+c1}{\PYGZsh{} Install xeus\PYGZhy{}lua from source}
git\PYG{+w}{ }clone\PYG{+w}{ }git@github.com:jupyter\PYGZhy{}xeus/xeus\PYGZhy{}lua.git
\PYG{n+nb}{cd}\PYG{+w}{ }xeus\PYGZhy{}lua\PYG{+w}{ }\PYG{o}{\PYGZam{}\PYGZam{}}\PYG{+w}{ }mkdir\PYG{+w}{ }build\PYG{+w}{ }\PYG{o}{\PYGZam{}\PYGZam{}}\PYG{+w}{ }\PYG{n+nb}{cd}\PYG{+w}{ }build
cmake\PYG{+w}{ }..\PYG{+w}{ }\PYGZhy{}D\PYG{+w}{ }\PYG{n+nv}{CMAKE\PYGZus{}PREFIX\PYGZus{}PATH}\PYG{o}{=}\PYG{n+nv}{\PYGZdl{}CONDA\PYGZus{}PREFIX}\PYG{+w}{ }\PYGZhy{}D\PYG{+w}{ }\PYG{n+nv}{CMAKE\PYGZus{}INSTALL\PYGZus{}PREFIX}\PYG{o}{=}\PYG{n+nv}{\PYGZdl{}CONDA\PYGZus{}PREFIX}\PYG{+w}{ }\PYGZhy{}D\PYG{+w}{ }\PYG{n+nv}{CMAKE\PYGZus{}INSTALL\PYGZus{}LIBDIR}\PYG{o}{=}lib
\end{sphinxVerbatim}

\sphinxAtStartPar
First, we call \sphinxcode{\sphinxupquote{01\sphinxhyphen{}create\sphinxhyphen{}env.sh}} with the parameters needed. This means that both files will using the same parameters. Once this is done, we must source \sphinxcode{\sphinxupquote{miniconda}} such that we can activate the environment, use \sphinxcode{\sphinxupquote{conda}} to install \sphinxcode{\sphinxupquote{ipykernel}}, and link the environment to the kernel with the same name of the environment.


\section{Running the Script}
\label{\detokenize{notebooks/01-automating-setup:running-the-script}}
\sphinxAtStartPar
We can set permissions and run the script with the following where \sphinxcode{\sphinxupquote{\textless{}PROJECT\_PREFIX\textgreater{}}} is the \sphinxcode{\sphinxupquote{pwd}}:

\begin{sphinxVerbatim}[commandchars=\\\{\}]
chmod\PYG{+w}{ }\PYGZhy{}R\PYG{+w}{ }\PYG{l+m}{755}\PYG{+w}{ }scripts
./scripts/02\PYGZhy{}build\PYGZhy{}env.sh\PYG{+w}{ }\PYGZlt{}PROJECT\PYGZus{}PREFIX\PYGZgt{}
\end{sphinxVerbatim}

\sphinxAtStartPar
This will create a new \sphinxcode{\sphinxupquote{conda}} environment called \sphinxcode{\sphinxupquote{jupyter\sphinxhyphen{}lua}}. Furthermore, as we saw in \sphinxcode{\sphinxupquote{02\sphinxhyphen{}build\sphinxhyphen{}env.sh}}, it will also create a Jupyter kernel that is called \sphinxcode{\sphinxupquote{jupyter\sphinxhyphen{}lua}}, which we can begin to use immediately after starting up Jupyter Notebooks.


\chapter{Next Steps}
\label{\detokenize{notebooks/01-automating-setup:next-steps}}
\sphinxAtStartPar
We will return to these build scripts once we configure and build our own kernel with expanded features to better suit the project.

\sphinxstepscope

\begin{sphinxVerbatim}[commandchars=\\\{\}]
\PYG{n+nt}{title}\PYG{p}{:}\PYG{+w}{ }\PYG{l+lScalar+lScalarPlain}{MyST}\PYG{l+lScalar+lScalarPlain}{ }\PYG{l+lScalar+lScalarPlain}{Markdown}\PYG{l+lScalar+lScalarPlain}{ }\PYG{l+lScalar+lScalarPlain}{Integration}\PYG{l+lScalar+lScalarPlain}{ }\PYG{l+lScalar+lScalarPlain}{for}\PYG{l+lScalar+lScalarPlain}{ }\PYG{l+lScalar+lScalarPlain}{Jupyter}\PYG{l+lScalar+lScalarPlain}{ }\PYG{l+lScalar+lScalarPlain}{Notebooks}
\PYG{n+nt}{description}\PYG{p}{:}\PYG{+w}{ }\PYG{l+lScalar+lScalarPlain}{This}\PYG{l+lScalar+lScalarPlain}{ }\PYG{l+lScalar+lScalarPlain}{notebook}\PYG{l+lScalar+lScalarPlain}{ }\PYG{l+lScalar+lScalarPlain}{explores}\PYG{l+lScalar+lScalarPlain}{ }\PYG{l+lScalar+lScalarPlain}{the}\PYG{l+lScalar+lScalarPlain}{ }\PYG{l+lScalar+lScalarPlain}{conventions}\PYG{l+lScalar+lScalarPlain}{ }\PYG{l+lScalar+lScalarPlain}{for}\PYG{l+lScalar+lScalarPlain}{ }\PYG{l+lScalar+lScalarPlain}{yaml}\PYG{l+lScalar+lScalarPlain}{ }\PYG{l+lScalar+lScalarPlain}{frontmatter}\PYG{l+lScalar+lScalarPlain}{ }\PYG{l+lScalar+lScalarPlain}{in}\PYG{l+lScalar+lScalarPlain}{ }\PYG{l+lScalar+lScalarPlain}{MyST}\PYG{l+lScalar+lScalarPlain}{ }\PYG{l+lScalar+lScalarPlain}{Formatted}\PYG{l+lScalar+lScalarPlain}{ }\PYG{l+lScalar+lScalarPlain}{notebooks}\PYG{l+lScalar+lScalarPlain}{ }\PYG{l+lScalar+lScalarPlain}{and}\PYG{l+lScalar+lScalarPlain}{ }\PYG{l+lScalar+lScalarPlain}{some}\PYG{l+lScalar+lScalarPlain}{ }\PYG{l+lScalar+lScalarPlain}{of}\PYG{l+lScalar+lScalarPlain}{ }\PYG{l+lScalar+lScalarPlain}{the}\PYG{l+lScalar+lScalarPlain}{ }\PYG{l+lScalar+lScalarPlain}{tool\PYGZsq{}s}\PYG{l+lScalar+lScalarPlain}{ }\PYG{l+lScalar+lScalarPlain}{other}\PYG{l+lScalar+lScalarPlain}{ }\PYG{l+lScalar+lScalarPlain}{features.}
\PYG{n+nt}{date}\PYG{p}{:}\PYG{+w}{ }\PYG{l+lScalar+lScalarPlain}{2024\PYGZhy{}01\PYGZhy{}04}
\end{sphinxVerbatim}

\begin{sphinxuseclass}{cell}
\begin{sphinxuseclass}{cell_input}
\begin{sphinxVerbatim}[commandchars=\\\{\}]
\PYG{o}{\PYGZpc{}\PYGZpc{}html}
\PYG{p}{\PYGZlt{}}\PYG{n+nt}{style}\PYG{p}{\PYGZgt{}}
\PYG{+w}{    }\PYG{n+nt}{body}\PYG{+w}{ }\PYG{p}{\PYGZob{}}
\PYG{+w}{        }\PYG{c}{/* change \PYGZhy{}\PYGZhy{}vscode\PYGZhy{}font\PYGZhy{}family if it\PYGZsq{}s different on your system */}
\PYG{+w}{        }\PYG{n+nv}{\PYGZhy{}\PYGZhy{}vscode\PYGZhy{}font\PYGZhy{}family}\PYG{p}{:}\PYG{+w}{ }\PYG{l+s+s2}{\PYGZdq{}lmroman17\PYGZhy{}regular\PYGZdq{}}
\PYG{+w}{    }\PYG{p}{\PYGZcb{}}
\PYG{p}{\PYGZlt{}}\PYG{p}{/}\PYG{n+nt}{style}\PYG{p}{\PYGZgt{}}
\end{sphinxVerbatim}

\end{sphinxuseclass}
\begin{sphinxuseclass}{cell_output}
\begin{sphinxVerbatim}[commandchars=\\\{\}]
\PYGZlt{}IPython.core.display.HTML object\PYGZgt{}
\end{sphinxVerbatim}

\end{sphinxuseclass}
\end{sphinxuseclass}

\chapter{MyST Markdown and Jupyter Notebook}
\label{\detokenize{notebooks/02-myst.integration:myst-markdown-and-jupyter-notebook}}\label{\detokenize{notebooks/02-myst.integration::doc}}
\sphinxAtStartPar
\sphinxhref{https://myst-nb.readthedocs.io/en/latest/}{MyST\sphinxhyphen{}NB} is a Spinx and Docutils extension for compiling Jupyter Notebooks into high quality documentation formats.

\sphinxAtStartPar
This extension will allow us to mix Jupyter notebooks with text\sphinxhyphen{}based notebooks, Markdown, and RST documents. Using MyST flavored Markdown syntax \textendash{}which extends \sphinxhref{https://commonmark.org/}{CommonMark Markdown} language including admonitions and \sphinxhref{https://myst-parser.readthedocs.io/en/latest/syntax/typography.html}{additional syntax}\textendash{} we can author technical and scientific documentation through our regular Jupyter interface and render beautiful PDFs, MS Word Documents, LaTeX, and more.

\sphinxAtStartPar
I am really excited by using MyST as a potential substitute for my original \sphinxcode{\sphinxupquote{Pandoc}} configuration project and want to turn this Jupyter setup into it’s own separate build script so that it can be used added to future Jupyter environments.

\sphinxAtStartPar
:::\{important\} Objectives
An important objective of this Jupyter Notebook is to understand the implementation of MyST and it’s tools to generate better notebooks that can be converted to publish\sphinxhyphen{}quality products.
:::


\chapter{Getting Started with MyST \& Sphinx}
\label{\detokenize{notebooks/02-myst.integration:getting-started-with-myst-sphinx}}

\chapter{Frontmatter}
\label{\detokenize{notebooks/02-myst.integration:frontmatter}}
\sphinxAtStartPar
Frontmatter allows you to specify metadata and options about how your project should behave or render. Included in frontmatter are things like the document or project \sphinxcode{\sphinxupquote{title}}, what \sphinxcode{\sphinxupquote{thumbnail}} to use for sit or content previews, \sphinxcode{\sphinxupquote{authors}} that contributed to the work, and scientific identifiers like a \sphinxcode{\sphinxupquote{doi}}. Adding frontmatter ensures that these properties are available to downstream tools or build processes like building \sphinxstylestrong{Scientific PDFs}.


\section{Where to set frontmatter}
\label{\detokenize{notebooks/02-myst.integration:where-to-set-frontmatter}}
\sphinxAtStartPar
Frontmatter can be set in a markdown (\sphinxcode{\sphinxupquote{md}}) or notebook (\sphinxcode{\sphinxupquote{ipynb}}) file (described as a “page” below) or in the \sphinxcode{\sphinxupquote{project:}} section of a \sphinxcode{\sphinxupquote{myst.yml}} file. When project frontmatter is set in a \sphinxcode{\sphinxupquote{myst.yml}} file, those settings will be applied to all content in that project (apart from “page only” fields).


\subsection{In a MyST markdown file}
\label{\detokenize{notebooks/02-myst.integration:in-a-myst-markdown-file}}
\sphinxAtStartPar
A frontmatter section can be added at the top of any \sphinxcode{\sphinxupquote{md}} file using \sphinxcode{\sphinxupquote{\sphinxhyphen{}\sphinxhyphen{}\sphinxhyphen{}}} delimiters.

\begin{sphinxVerbatim}[commandchars=\\\{\}]
\PYG{n+nt}{title}\PYG{p}{:}\PYG{+w}{ }\PYG{l+lScalar+lScalarPlain}{My}\PYG{l+lScalar+lScalarPlain}{ }\PYG{l+lScalar+lScalarPlain}{First}\PYG{l+lScalar+lScalarPlain}{ }\PYG{l+lScalar+lScalarPlain}{Article}
\PYG{n+nt}{date}\PYG{p}{:}\PYG{+w}{ }\PYG{l+lScalar+lScalarPlain}{2024\PYGZhy{}01\PYGZhy{}04}
\PYG{n+nt}{authors}\PYG{p}{:}
\PYG{+w}{    }\PYG{p+pIndicator}{\PYGZhy{}}\PYG{+w}{ }\PYG{l+lScalar+lScalarPlain}{Nyki}\PYG{l+lScalar+lScalarPlain}{ }\PYG{l+lScalar+lScalarPlain}{Anderson}
\PYG{+w}{      }\PYG{l+lScalar+lScalarPlain}{affiliations}\PYG{p+pIndicator}{:}
\PYG{+w}{        }\PYG{p+pIndicator}{\PYGZhy{}}\PYG{+w}{ }\PYG{l+lScalar+lScalarPlain}{University}\PYG{l+lScalar+lScalarPlain}{ }\PYG{l+lScalar+lScalarPlain}{of}\PYG{l+lScalar+lScalarPlain}{ }\PYG{l+lScalar+lScalarPlain}{Europe}
\end{sphinxVerbatim}


\subsection{In a Jupyter Notebook}
\label{\detokenize{notebooks/02-myst.integration:in-a-jupyter-notebook}}
\sphinxAtStartPar
Frontmatter can be added to the first cell of a Jupyter Notebook, that cell should be a Markdown cell and use \sphinxcode{\sphinxupquote{\sphinxhyphen{}\sphinxhyphen{}\sphinxhyphen{}}} delimiters as above.
\begin{quote}

\sphinxAtStartPar
\sphinxstylestrong{Hint!} Install JupyterLab Myst
To have properly formatted frontmatter, you can install the \sphinxcode{\sphinxupquote{jupyterlab\sphinxhyphen{}myst}} plugin for Jupyter.

\sphinxAtStartPar
\sphinxcode{\sphinxupquote{pip install jupyterlab\_myst}}

\sphinxAtStartPar
Without the extension installed, remember to format the contents of the section as valid \sphinxcode{\sphinxupquote{yaml}} even though when rendered, the cell will not look well formatted in your notebook.
\end{quote}


\subsection{In a \sphinxstyleliteralintitle{\sphinxupquote{myst.yml}} file}
\label{\detokenize{notebooks/02-myst.integration:in-a-myst-yml-file}}
\sphinxAtStartPar
Frontmatter fields can be added directly to any \sphinxcode{\sphinxupquote{project:}} section within a \sphinxcode{\sphinxupquote{myst.yml}} file. If your root \sphinxcode{\sphinxupquote{myst.yml}} file only contains a \sphinxcode{\sphinxupquote{site:}} section, and you want to add frontmatter, add a \sphinxcode{\sphinxupquote{project:}} section at the top level and add the fields there, e.g.,

\begin{sphinxVerbatim}[commandchars=\\\{\}]
\PYG{n+nt}{myst}\PYG{p}{:}\PYG{+w}{ }\PYG{l+lScalar+lScalarPlain}{v1}
\PYG{n+nt}{site}\PYG{p}{:}\PYG{+w}{ }
\PYG{n+nt}{project}\PYG{p}{:}
\PYG{+w}{    }\PYG{n+nt}{license}\PYG{p}{:}\PYG{+w}{ }\PYG{l+lScalar+lScalarPlain}{CC\PYGZhy{}BY\PYGZhy{}4.0}
\PYG{+w}{    }\PYG{n+nt}{open\PYGZus{}access}\PYG{p}{:}\PYG{+w}{ }\PYG{l+lScalar+lScalarPlain}{true}
\end{sphinxVerbatim}


\section{Available frontmatter fields}
\label{\detokenize{notebooks/02-myst.integration:available-frontmatter-fields}}
\sphinxAtStartPar
The following table lists the available frontmatter fields, a brief description and a note on how the field behaves depending on whether it is set on a page or at the project level. Where a field itself is an object with sub\sphinxhyphen{}fields, see the relevant description on the page below.
\begin{itemize}
\item {} 
\sphinxAtStartPar
\sphinxcode{\sphinxupquote{title}} \sphinxhyphen{} a string (page \& project)

\item {} 
\sphinxAtStartPar
\sphinxcode{\sphinxupquote{description}} \sphinxhyphen{} a string (page \& project)

\item {} 
\sphinxAtStartPar
\sphinxcode{\sphinxupquote{short\_title}} \sphinxhyphen{} a string (page \& project)

\item {} 
\sphinxAtStartPar
\sphinxcode{\sphinxupquote{name}} \sphinxhyphen{} a string (page \& project)

\item {} 
\sphinxAtStartPar
\sphinxcode{\sphinxupquote{tags}} \sphinxhyphen{} a list of strings (page only)

\item {} 
\sphinxAtStartPar
\sphinxcode{\sphinxupquote{thumbnail}} \sphinxhyphen{} a link to a local or remote image (page only)

\item {} 
\sphinxAtStartPar
\sphinxcode{\sphinxupquote{subtitle}} \sphinxhyphen{} a string (page only)

\item {} 
\sphinxAtStartPar
\sphinxcode{\sphinxupquote{date}} \sphinxhyphen{} a valid date (page can override project)

\item {} 
\sphinxAtStartPar
\sphinxcode{\sphinxupquote{authors}} \sphinxhyphen{} a list of author objects (page can override project)

\item {} 
\sphinxAtStartPar
\sphinxcode{\sphinxupquote{affiliations}} \sphinxhyphen{} a list of affiliation objects (page can override project)

\item {} 
\sphinxAtStartPar
\sphinxcode{\sphinxupquote{doi}} \sphinxhyphen{} a valid DOI, either URL or id (page can override project)

\item {} 
\sphinxAtStartPar
\sphinxcode{\sphinxupquote{arxiv}} \sphinxhyphen{} a valid arXiv reference, either URL or id (page can override project)

\item {} 
\sphinxAtStartPar
\sphinxcode{\sphinxupquote{open\_access}} \sphinxhyphen{} boolean (page can override project)

\item {} 
\sphinxAtStartPar
\sphinxcode{\sphinxupquote{license}} \sphinxhyphen{} a license object or a string (page can override project)

\item {} 
\sphinxAtStartPar
\sphinxcode{\sphinxupquote{funding}} \sphinxhyphen{} a funding object (page can override project)

\item {} 
\sphinxAtStartPar
\sphinxcode{\sphinxupquote{github}} \sphinxhyphen{} a valid GitHub URL or \sphinxcode{\sphinxupquote{owner/reponame}} (page can override project)

\item {} 
\sphinxAtStartPar
\sphinxcode{\sphinxupquote{binder}} any valid URL (page can override project)

\item {} 
\sphinxAtStartPar
\sphinxcode{\sphinxupquote{subject}} \sphinxhyphen{} a string (page can override project)

\item {} 
\sphinxAtStartPar
\sphinxcode{\sphinxupquote{venue}} \sphinxhyphen{} a venue object (page can override project)

\item {} 
\sphinxAtStartPar
\sphinxcode{\sphinxupquote{biblio}} \sphinxhyphen{} a biblio object with various fields (page can override project)

\item {} 
\sphinxAtStartPar
\sphinxcode{\sphinxupquote{math}} \sphinxhyphen{} a dictionary of math macros (page can override project)

\item {} 
\sphinxAtStartPar
\sphinxcode{\sphinxupquote{abbreviations}} \sphinxhyphen{} a dictionary of abbreviations in the project (page can override project)

\item {} 
\sphinxAtStartPar
\sphinxcode{\sphinxupquote{parts}} \sphinxhyphen{} a dictionary of arbitrary content parts, not part of the main article, for example \sphinxcode{\sphinxupquote{abstract}}, \sphinxcode{\sphinxupquote{data\_availability}} (page only)

\item {} 
\sphinxAtStartPar
\sphinxcode{\sphinxupquote{options}} \sphinxhyphen{} a dictionary of arbitrary options validated and consumed by templates, for example, during site or PDF build (page can override project)

\end{itemize}


\section{Field Behavior}
\label{\detokenize{notebooks/02-myst.integration:field-behavior}}
\sphinxAtStartPar
Frontmatter can be attached to a “page”, meaning a local \sphinxcode{\sphinxupquote{.md}} or \sphinxcode{\sphinxupquote{.ipynb}} or a “project”. However, individual frontmatter fields are not uniformly available at both levels, and certain behavior of certain fields are different between project and page levels. There are three field behaviors to be aware of:
\begin{itemize}
\item {} 
\sphinxAtStartPar
\sphinxcode{\sphinxupquote{page \& project}} : the field is available on both th epage \& project but they are independent

\item {} 
\sphinxAtStartPar
\sphinxcode{\sphinxupquote{page only}} : the field is only available on pages, and not present on projects and it will be ignored if set there.

\item {} 
\sphinxAtStartPar
\sphinxcode{\sphinxupquote{page can override project}} : the field is available on both page \& project but the value of the field on the page will override any set of the project. Note that the page field must be omitted or undefined, for the project value to be used, value of \sphinxcode{\sphinxupquote{null}} (or \sphinxcode{\sphinxupquote{{[}{]}}} in the case of \sphinxcode{\sphinxupquote{authors}}) will still override the project value but clear the field for that page.

\end{itemize}


\section{Thumbnail \& Banner}
\label{\detokenize{notebooks/02-myst.integration:thumbnail-banner}}
\sphinxAtStartPar
The thumbnail is used in previews for your site in applications like Twitter, Slack, or any other link preview service. This should, by convention, be included in a \sphinxcode{\sphinxupquote{thumbnails}} folder next to your content. You can also explicitly set this field to any other image on your local file system or a remote URL to an image. This image will get copied over to your public folder and optimized when you build your project.

\begin{sphinxVerbatim}[commandchars=\\\{\}]
\PYG{n+nt}{thumbnail}\PYG{p}{:}\PYG{+w}{ }\PYG{l+lScalar+lScalarPlain}{thumbnails/myThumbnail.png}
\end{sphinxVerbatim}

\sphinxAtStartPar
If you do not specify an image the first image in the content of a page will be selected. If you explicitly do not want an image, set \sphinxcode{\sphinxupquote{thumbnail}} to \sphinxcode{\sphinxupquote{null}}.

\sphinxAtStartPar
You can also set a banner image which will show up in certain themes, for example, the \sphinxcode{\sphinxupquote{article\sphinxhyphen{}theme}}:

\begin{sphinxVerbatim}[commandchars=\\\{\}]
\PYG{n+nt}{banner}\PYG{p}{:}\PYG{+w}{ }\PYG{l+lScalar+lScalarPlain}{banner.png}
\end{sphinxVerbatim}

\sphinxAtStartPar
\sphinxincludegraphics[width=1872\sphinxpxdimen,height=1338\sphinxpxdimen]{{banner}.jpg}


\section{Authors}
\label{\detokenize{notebooks/02-myst.integration:authors}}
\sphinxAtStartPar
The \sphinxcode{\sphinxupquote{authors}} field is a list of \sphinxcode{\sphinxupquote{author}} objects. Available fields in the author object are:
\begin{itemize}
\item {} 
\sphinxAtStartPar
\sphinxcode{\sphinxupquote{name}} : a string OR CSL\sphinxhyphen{}JSON author object \sphinxhyphen{} the author’s full name; if a string, this will be parsed automatically. Otherwise, the object may contain \sphinxcode{\sphinxupquote{given}}, \sphinxcode{\sphinxupquote{surname}}, \sphinxcode{\sphinxupquote{non\_dropping\_particle}}, \sphinxcode{\sphinxupquote{dropping\_particle}}, \sphinxcode{\sphinxupquote{suffix}}, and full name \sphinxcode{\sphinxupquote{literal}}.

\item {} 
\sphinxAtStartPar
\sphinxcode{\sphinxupquote{orcid}} : a string \sphinxhyphen{} a valid ORCID identifier with or without the URL.

\item {} 
\sphinxAtStartPar
\sphinxcode{\sphinxupquote{corresponding}} : boolean \sphinxhyphen{} flags any corresponding authors, you must include an \sphinxcode{\sphinxupquote{email}} if true.

\item {} 
\sphinxAtStartPar
\sphinxcode{\sphinxupquote{url}} : a string \sphinxhyphen{} website or homepage of the author.

\item {} 
\sphinxAtStartPar
\sphinxcode{\sphinxupquote{roles}} : a list of strings \sphinxhyphen{} must be valid \sphinxstylestrong{CRediT Contributor Roles}.

\end{itemize}

\begin{sphinxVerbatim}[commandchars=\\\{\}]
\PYG{n+nt}{authors}\PYG{p}{:}
\PYG{+w}{    }\PYG{p+pIndicator}{\PYGZhy{}}\PYG{+w}{ }\PYG{n+nt}{name}\PYG{p}{:}\PYG{+w}{ }\PYG{l+lScalar+lScalarPlain}{Nyki}\PYG{l+lScalar+lScalarPlain}{ }\PYG{l+lScalar+lScalarPlain}{Anderson}
\PYG{+w}{      }\PYG{n+nt}{roles}\PYG{p}{:}
\PYG{+w}{        }\PYG{p+pIndicator}{\PYGZhy{}}\PYG{+w}{ }\PYG{l+lScalar+lScalarPlain}{Conceptualization}
\PYG{+w}{        }\PYG{p+pIndicator}{\PYGZhy{}}\PYG{+w}{ }\PYG{l+lScalar+lScalarPlain}{Data}\PYG{l+lScalar+lScalarPlain}{ }\PYG{l+lScalar+lScalarPlain}{curation}
\PYG{+w}{        }\PYG{p+pIndicator}{\PYGZhy{}}\PYG{+w}{ }\PYG{l+lScalar+lScalarPlain}{Validation}
\end{sphinxVerbatim}
\begin{quote}

\sphinxAtStartPar
\sphinxstylestrong{CRediT Roles}

\sphinxAtStartPar
There are 14 official contributor roles that are in the NISO CRediT Role standard. In addition to British english, incorrect case or punctuation, there are also a number of aliases that can be used for various roles.
\begin{itemize}
\item {} 
\sphinxAtStartPar
Conceptualization (alias: conceptualisation)

\item {} 
\sphinxAtStartPar
Data curation

\item {} 
\sphinxAtStartPar
Formal analysis (alias: analysis)

\item {} 
\sphinxAtStartPar
Funding acquisition

\item {} 
\sphinxAtStartPar
Investigation

\item {} 
\sphinxAtStartPar
Methodology

\item {} 
\sphinxAtStartPar
Project administration (alias: administration)

\item {} 
\sphinxAtStartPar
Resources

\item {} 
\sphinxAtStartPar
Software

\item {} 
\sphinxAtStartPar
Supervision

\item {} 
\sphinxAtStartPar
Validation

\item {} 
\sphinxAtStartPar
Visualization (alias: visualisation)

\item {} 
\sphinxAtStartPar
Writing \sphinxhyphen{} original draft (alias: writing)

\item {} 
\sphinxAtStartPar
Writing \sphinxhyphen{} review \& editing (alias: editing, review)

\end{itemize}
\end{quote}
\begin{itemize}
\item {} 
\sphinxAtStartPar
\sphinxcode{\sphinxupquote{affiliations}} : a list of strings that identify or create an affiliation or a full \sphinxcode{\sphinxupquote{Affiliation}} object, for example:

\end{itemize}

\begin{sphinxVerbatim}[commandchars=\\\{\}]
\PYG{n+nt}{authors}\PYG{p}{:}
\PYG{+w}{    }\PYG{p+pIndicator}{\PYGZhy{}}\PYG{+w}{ }\PYG{n+nt}{name}\PYG{p}{:}\PYG{+w}{ }\PYG{l+lScalar+lScalarPlain}{Nyki}\PYG{l+lScalar+lScalarPlain}{ }\PYG{l+lScalar+lScalarPlain}{Anderson}
\PYG{+w}{      }\PYG{n+nt}{affiliations}\PYG{p}{:}
\PYG{+w}{        }\PYG{p+pIndicator}{\PYGZhy{}}\PYG{+w}{ }\PYG{n+nt}{id}\PYG{p}{:}\PYG{+w}{ }\PYG{l+lScalar+lScalarPlain}{ubc}
\PYG{+w}{          }\PYG{n+nt}{institution}\PYG{p}{:}\PYG{+w}{ }\PYG{l+lScalar+lScalarPlain}{University}\PYG{l+lScalar+lScalarPlain}{ }\PYG{l+lScalar+lScalarPlain}{of}\PYG{l+lScalar+lScalarPlain}{ }\PYG{l+lScalar+lScalarPlain}{British}\PYG{l+lScalar+lScalarPlain}{ }\PYG{l+lScalar+lScalarPlain}{Columbia}
\PYG{+w}{          }\PYG{n+nt}{ror}\PYG{p}{:}\PYG{+w}{ }\PYG{l+lScalar+lScalarPlain}{034rmrcq20}
\PYG{+w}{          }\PYG{n+nt}{department}\PYG{p}{:}\PYG{+w}{ }\PYG{l+lScalar+lScalarPlain}{Earth,}\PYG{l+lScalar+lScalarPlain}{ }\PYG{l+lScalar+lScalarPlain}{Ocean}\PYG{l+lScalar+lScalarPlain}{ }\PYG{l+lScalar+lScalarPlain}{and}\PYG{l+lScalar+lScalarPlain}{ }\PYG{l+lScalar+lScalarPlain}{Atmospheric}\PYG{l+lScalar+lScalarPlain}{ }\PYG{l+lScalar+lScalarPlain}{Sciences}
\PYG{+w}{        }\PYG{p+pIndicator}{\PYGZhy{}}\PYG{+w}{ }\PYG{l+lScalar+lScalarPlain}{ACME}\PYG{l+lScalar+lScalarPlain}{ }\PYG{l+lScalar+lScalarPlain}{Inc}
\PYG{+w}{    }\PYG{p+pIndicator}{\PYGZhy{}}\PYG{+w}{ }\PYG{n+nt}{name}\PYG{p}{:}\PYG{+w}{ }\PYG{l+lScalar+lScalarPlain}{Julian}\PYG{l+lScalar+lScalarPlain}{ }\PYG{l+lScalar+lScalarPlain}{Todman}
\PYG{+w}{      }\PYG{n+nt}{affiliation}\PYG{p}{:}\PYG{+w}{ }\PYG{l+lScalar+lScalarPlain}{ubc}\PYG{+w}{   }
\end{sphinxVerbatim}

\sphinxAtStartPar
See \sphinxhref{https://mystmd.org/guide/frontmatter\#affiliations}{Affiliations} for more information on how to concisely write affiliations.
\begin{itemize}
\item {} 
\sphinxAtStartPar
\sphinxcode{\sphinxupquote{equal\sphinxhyphen{}contributor}} : a boolean, indicates that the author is an equal contributor.

\item {} 
\sphinxAtStartPar
\sphinxcode{\sphinxupquote{deceased}} : a boolean, indicates that the author is a deceased.

\item {} 
\sphinxAtStartPar
\sphinxcode{\sphinxupquote{twitter}} : a twitter username.

\item {} 
\sphinxAtStartPar
\sphinxcode{\sphinxupquote{github}} : a GitHub username.

\item {} 
\sphinxAtStartPar
\sphinxcode{\sphinxupquote{note}} : a string, a freeform field to indicate additional information about the author, for example, acknowledgments or specific correspondence information.

\item {} 
\sphinxAtStartPar
\sphinxcode{\sphinxupquote{phone}} : a phone number, e.g., (301) 754 \sphinxhyphen{} 5766.

\item {} 
\sphinxAtStartPar
\sphinxcode{\sphinxupquote{fax}} : for people who still use these machines.

\end{itemize}


\section{Affiliations}
\label{\detokenize{notebooks/02-myst.integration:affiliations}}
\sphinxAtStartPar
You can create an affiliation directly by adding it to an author, and it can be as simple as a single string.

\begin{sphinxVerbatim}[commandchars=\\\{\}]
\PYG{n+nt}{authors}\PYG{p}{:}
\PYG{+w}{    }\PYG{p+pIndicator}{\PYGZhy{}}\PYG{+w}{ }\PYG{n+nt}{name}\PYG{p}{:}\PYG{+w}{ }\PYG{l+lScalar+lScalarPlain}{Nyki}\PYG{l+lScalar+lScalarPlain}{ }\PYG{l+lScalar+lScalarPlain}{Anderson}
\PYG{+w}{      }\PYG{n+nt}{affiliation}\PYG{p}{:}\PYG{+w}{ }\PYG{l+lScalar+lScalarPlain}{University}\PYG{l+lScalar+lScalarPlain}{ }\PYG{l+lScalar+lScalarPlain}{of}\PYG{l+lScalar+lScalarPlain}{ }\PYG{l+lScalar+lScalarPlain}{British}\PYG{l+lScalar+lScalarPlain}{ }\PYG{l+lScalar+lScalarPlain}{Columbia}
\end{sphinxVerbatim}

\sphinxAtStartPar
You can also add much more information to any affiliation, such as a ROR, ISNI, or an address. A very complete affiliations list for an author at the University of British Columbia is:

\begin{sphinxVerbatim}[commandchars=\\\{\}]
\PYG{n+nt}{authors}\PYG{p}{:}
\PYG{+w}{    }\PYG{p+pIndicator}{\PYGZhy{}}\PYG{+w}{ }\PYG{n+nt}{name}\PYG{p}{:}\PYG{+w}{ }\PYG{l+lScalar+lScalarPlain}{Nyki}\PYG{l+lScalar+lScalarPlain}{ }\PYG{l+lScalar+lScalarPlain}{Anderson}
\PYG{+w}{      }\PYG{n+nt}{affiliations}\PYG{p}{:}
\PYG{+w}{        }\PYG{p+pIndicator}{\PYGZhy{}}\PYG{+w}{ }\PYG{n+nt}{id}\PYG{p}{:}\PYG{+w}{ }\PYG{l+lScalar+lScalarPlain}{ubc}
\PYG{+w}{          }\PYG{n+nt}{institution}\PYG{p}{:}\PYG{+w}{ }\PYG{l+lScalar+lScalarPlain}{University}\PYG{l+lScalar+lScalarPlain}{ }\PYG{l+lScalar+lScalarPlain}{of}\PYG{l+lScalar+lScalarPlain}{ }\PYG{l+lScalar+lScalarPlain}{British}\PYG{l+lScalar+lScalarPlain}{ }\PYG{l+lScalar+lScalarPlain}{Columbia}
\PYG{+w}{          }\PYG{n+nt}{ror}\PYG{p}{:}\PYG{+w}{ }\PYG{l+lScalar+lScalarPlain}{https://ror.org/03rmrcq20}
\PYG{+w}{          }\PYG{n+nt}{isni}\PYG{p}{:}\PYG{+w}{ }\PYG{l+lScalar+lScalarPlain}{0000}\PYG{l+lScalar+lScalarPlain}{ }\PYG{l+lScalar+lScalarPlain}{0001}\PYG{l+lScalar+lScalarPlain}{ }\PYG{l+lScalar+lScalarPlain}{2288}\PYG{l+lScalar+lScalarPlain}{ }\PYG{l+lScalar+lScalarPlain}{9830}
\PYG{+w}{          }\PYG{n+nt}{department}\PYG{p}{:}\PYG{+w}{ }\PYG{l+lScalar+lScalarPlain}{Department}\PYG{l+lScalar+lScalarPlain}{ }\PYG{l+lScalar+lScalarPlain}{of}\PYG{l+lScalar+lScalarPlain}{ }\PYG{l+lScalar+lScalarPlain}{Earth,}\PYG{l+lScalar+lScalarPlain}{ }\PYG{l+lScalar+lScalarPlain}{Ocean}\PYG{l+lScalar+lScalarPlain}{ }\PYG{l+lScalar+lScalarPlain}{and}\PYG{l+lScalar+lScalarPlain}{ }\PYG{l+lScalar+lScalarPlain}{Atmospheric}\PYG{l+lScalar+lScalarPlain}{ }\PYG{l+lScalar+lScalarPlain}{Sciences}
\PYG{+w}{          }\PYG{n+nt}{address}\PYG{p}{:}\PYG{+w}{ }\PYG{l+lScalar+lScalarPlain}{2020}\PYG{l+lScalar+lScalarPlain}{ }\PYG{l+lScalar+lScalarPlain}{\PYGZhy{}}\PYG{l+lScalar+lScalarPlain}{ }\PYG{l+lScalar+lScalarPlain}{2207}\PYG{l+lScalar+lScalarPlain}{ }\PYG{l+lScalar+lScalarPlain}{Main}\PYG{l+lScalar+lScalarPlain}{ }\PYG{l+lScalar+lScalarPlain}{Mall}
\PYG{+w}{          }\PYG{n+nt}{city}\PYG{p}{:}\PYG{+w}{ }\PYG{l+lScalar+lScalarPlain}{Vancouver}
\PYG{+w}{          }\PYG{n+nt}{region}\PYG{p}{:}\PYG{+w}{ }\PYG{l+lScalar+lScalarPlain}{British}\PYG{l+lScalar+lScalarPlain}{ }\PYG{l+lScalar+lScalarPlain}{Columbia}
\PYG{+w}{          }\PYG{n+nt}{country}\PYG{p}{:}\PYG{+w}{ }\PYG{l+lScalar+lScalarPlain}{Canada}
\PYG{+w}{          }\PYG{n+nt}{postal\PYGZus{}code}\PYG{p}{:}\PYG{+w}{ }\PYG{l+lScalar+lScalarPlain}{V6T}\PYG{l+lScalar+lScalarPlain}{ }\PYG{l+lScalar+lScalarPlain}{1Z4}
\PYG{+w}{          }\PYG{n+nt}{phone}\PYG{p}{:}\PYG{+w}{ }\PYG{l+lScalar+lScalarPlain}{604}\PYG{l+lScalar+lScalarPlain}{ }\PYG{l+lScalar+lScalarPlain}{822}\PYG{l+lScalar+lScalarPlain}{ }\PYG{l+lScalar+lScalarPlain}{2449}
\PYG{+w}{    }\PYG{p+pIndicator}{\PYGZhy{}}\PYG{+w}{ }\PYG{n+nt}{name}\PYG{p}{:}\PYG{+w}{ }\PYG{l+lScalar+lScalarPlain}{Julian}\PYG{l+lScalar+lScalarPlain}{ }\PYG{l+lScalar+lScalarPlain}{Todman}
\PYG{+w}{      }\PYG{n+nt}{affiliation}\PYG{p}{:}\PYG{+w}{ }\PYG{l+lScalar+lScalarPlain}{ubc}\PYG{+w}{   }
\end{sphinxVerbatim}

\sphinxAtStartPar
Notice how you can use an \sphinxcode{\sphinxupquote{id}} to avoid writing this out for every coauthor. Additionally, if the affiliation is a single string and contains a semi\sphinxhyphen{}colon \sphinxcode{\sphinxupquote{;}} it will be treated as a list. The affiliations can also be added to your \sphinxcode{\sphinxupquote{project}} frontmatter in your \sphinxcode{\sphinxupquote{myst.yml}} and used across any document i the project.

\begin{sphinxVerbatim}[commandchars=\\\{\}]
\PYG{c+c1}{\PYGZsh{} article.md}
\PYG{n+nt}{title}\PYG{p}{:}\PYG{+w}{ }\PYG{l+lScalar+lScalarPlain}{My}\PYG{l+lScalar+lScalarPlain}{ }\PYG{l+lScalar+lScalarPlain}{Article}
\PYG{n+nt}{authors}\PYG{p}{:}
\PYG{+w}{  }\PYG{p+pIndicator}{\PYGZhy{}}\PYG{+w}{ }\PYG{n+nt}{name}\PYG{p}{:}\PYG{+w}{ }\PYG{l+lScalar+lScalarPlain}{Nyki}\PYG{l+lScalar+lScalarPlain}{ }\PYG{l+lScalar+lScalarPlain}{Anderson}
\PYG{+w}{    }\PYG{n+nt}{affiliation}\PYG{p}{:}\PYG{+w}{ }\PYG{l+lScalar+lScalarPlain}{ubc}
\PYG{+w}{  }\PYG{p+pIndicator}{\PYGZhy{}}\PYG{+w}{ }\PYG{n+nt}{name}\PYG{p}{:}\PYG{+w}{ }\PYG{l+lScalar+lScalarPlain}{Julian}\PYG{l+lScalar+lScalarPlain}{ }\PYG{l+lScalar+lScalarPlain}{Todman}
\PYG{+w}{    }\PYG{n+nt}{affiliations}\PYG{p}{:}\PYG{+w}{ }\PYG{l+lScalar+lScalarPlain}{ubc;}\PYG{l+lScalar+lScalarPlain}{ }\PYG{l+lScalar+lScalarPlain}{stanford}
\end{sphinxVerbatim}

\begin{sphinxVerbatim}[commandchars=\\\{\}]
\PYG{c+c1}{\PYGZsh{} myst.yml}
\PYG{n+nt}{affiliations}\PYG{p}{:}
\PYG{+w}{  }\PYG{p+pIndicator}{\PYGZhy{}}\PYG{+w}{ }\PYG{n+nt}{id}\PYG{p}{:}\PYG{+w}{ }\PYG{l+lScalar+lScalarPlain}{ubc}
\PYG{+w}{    }\PYG{n+nt}{institution}\PYG{p}{:}\PYG{+w}{ }\PYG{l+lScalar+lScalarPlain}{University}\PYG{l+lScalar+lScalarPlain}{ }\PYG{l+lScalar+lScalarPlain}{of}\PYG{l+lScalar+lScalarPlain}{ }\PYG{l+lScalar+lScalarPlain}{British}\PYG{l+lScalar+lScalarPlain}{ }\PYG{l+lScalar+lScalarPlain}{Columbia}
\PYG{+w}{    }\PYG{n+nt}{ror}\PYG{p}{:}\PYG{+w}{ }\PYG{l+lScalar+lScalarPlain}{https://ror.org/03rmrcq20}
\PYG{+w}{    }\PYG{n+nt}{isni}\PYG{p}{:}\PYG{+w}{ }\PYG{l+lScalar+lScalarPlain}{0000}\PYG{l+lScalar+lScalarPlain}{ }\PYG{l+lScalar+lScalarPlain}{0001}\PYG{l+lScalar+lScalarPlain}{ }\PYG{l+lScalar+lScalarPlain}{2288}\PYG{l+lScalar+lScalarPlain}{ }\PYG{l+lScalar+lScalarPlain}{9830}
\PYG{+w}{    }\PYG{n+nt}{department}\PYG{p}{:}\PYG{+w}{ }\PYG{l+lScalar+lScalarPlain}{Department}\PYG{l+lScalar+lScalarPlain}{ }\PYG{l+lScalar+lScalarPlain}{of}\PYG{l+lScalar+lScalarPlain}{ }\PYG{l+lScalar+lScalarPlain}{Earth,}\PYG{l+lScalar+lScalarPlain}{ }\PYG{l+lScalar+lScalarPlain}{Ocean}\PYG{l+lScalar+lScalarPlain}{ }\PYG{l+lScalar+lScalarPlain}{and}\PYG{l+lScalar+lScalarPlain}{ }\PYG{l+lScalar+lScalarPlain}{Atmospheric}\PYG{l+lScalar+lScalarPlain}{ }\PYG{l+lScalar+lScalarPlain}{Sciences}
\PYG{+w}{    }\PYG{n+nt}{address}\PYG{p}{:}\PYG{+w}{ }\PYG{l+lScalar+lScalarPlain}{2020}\PYG{l+lScalar+lScalarPlain}{ }\PYG{l+lScalar+lScalarPlain}{\textendash{}}\PYG{l+lScalar+lScalarPlain}{ }\PYG{l+lScalar+lScalarPlain}{2207}\PYG{l+lScalar+lScalarPlain}{ }\PYG{l+lScalar+lScalarPlain}{Main}\PYG{l+lScalar+lScalarPlain}{ }\PYG{l+lScalar+lScalarPlain}{Mall}
\PYG{+w}{    }\PYG{n+nt}{city}\PYG{p}{:}\PYG{+w}{ }\PYG{l+lScalar+lScalarPlain}{Vancouver}
\PYG{+w}{    }\PYG{n+nt}{region}\PYG{p}{:}\PYG{+w}{ }\PYG{l+lScalar+lScalarPlain}{British}\PYG{l+lScalar+lScalarPlain}{ }\PYG{l+lScalar+lScalarPlain}{Columbia}
\PYG{+w}{    }\PYG{n+nt}{country}\PYG{p}{:}\PYG{+w}{ }\PYG{l+lScalar+lScalarPlain}{Canada}
\PYG{+w}{    }\PYG{n+nt}{postal\PYGZus{}code}\PYG{p}{:}\PYG{+w}{ }\PYG{l+lScalar+lScalarPlain}{V6T}\PYG{l+lScalar+lScalarPlain}{ }\PYG{l+lScalar+lScalarPlain}{1Z4}
\PYG{+w}{    }\PYG{n+nt}{phone}\PYG{p}{:}\PYG{+w}{ }\PYG{l+lScalar+lScalarPlain}{604}\PYG{l+lScalar+lScalarPlain}{ }\PYG{l+lScalar+lScalarPlain}{822}\PYG{l+lScalar+lScalarPlain}{ }\PYG{l+lScalar+lScalarPlain}{2449}
\PYG{+w}{  }\PYG{p+pIndicator}{\PYGZhy{}}\PYG{+w}{ }\PYG{n+nt}{id}\PYG{p}{:}\PYG{+w}{ }\PYG{l+lScalar+lScalarPlain}{stanford}
\end{sphinxVerbatim}

\sphinxAtStartPar
If you use a string that is not recognized as an already defined affiliation in the project or article frontmatter, an affiliation will be created automatically and normalized so that it can be referenced:

\begin{sphinxVerbatim}[commandchars=\\\{\}]
\PYG{c+c1}{\PYGZsh{} Written Frontmatter}
\PYG{n+nt}{authors}\PYG{p}{:}
\PYG{+w}{  }\PYG{p+pIndicator}{\PYGZhy{}}\PYG{+w}{ }\PYG{n+nt}{name}\PYG{p}{:}\PYG{+w}{ }\PYG{l+lScalar+lScalarPlain}{Marissa}\PYG{l+lScalar+lScalarPlain}{ }\PYG{l+lScalar+lScalarPlain}{Myst}
\PYG{+w}{    }\PYG{n+nt}{affiliations}\PYG{p}{:}
\PYG{+w}{      }\PYG{p+pIndicator}{\PYGZhy{}}\PYG{+w}{ }\PYG{n+nt}{id}\PYG{p}{:}\PYG{+w}{ }\PYG{l+lScalar+lScalarPlain}{ubc}
\PYG{+w}{        }\PYG{n+nt}{institution}\PYG{p}{:}\PYG{+w}{ }\PYG{l+lScalar+lScalarPlain}{University}\PYG{l+lScalar+lScalarPlain}{ }\PYG{l+lScalar+lScalarPlain}{of}\PYG{l+lScalar+lScalarPlain}{ }\PYG{l+lScalar+lScalarPlain}{British}\PYG{l+lScalar+lScalarPlain}{ }\PYG{l+lScalar+lScalarPlain}{Columbia}
\PYG{+w}{        }\PYG{n+nt}{ror}\PYG{p}{:}\PYG{+w}{ }\PYG{l+lScalar+lScalarPlain}{03rmrcq20}
\PYG{+w}{        }\PYG{n+nt}{department}\PYG{p}{:}\PYG{+w}{ }\PYG{l+lScalar+lScalarPlain}{Earth,}\PYG{l+lScalar+lScalarPlain}{ }\PYG{l+lScalar+lScalarPlain}{Ocean}\PYG{l+lScalar+lScalarPlain}{ }\PYG{l+lScalar+lScalarPlain}{and}\PYG{l+lScalar+lScalarPlain}{ }\PYG{l+lScalar+lScalarPlain}{Atmospheric}\PYG{l+lScalar+lScalarPlain}{ }\PYG{l+lScalar+lScalarPlain}{Sciences}
\PYG{+w}{      }\PYG{p+pIndicator}{\PYGZhy{}}\PYG{+w}{ }\PYG{l+lScalar+lScalarPlain}{ACME}\PYG{l+lScalar+lScalarPlain}{ }\PYG{l+lScalar+lScalarPlain}{Inc}
\PYG{+w}{  }\PYG{p+pIndicator}{\PYGZhy{}}\PYG{+w}{ }\PYG{n+nt}{name}\PYG{p}{:}\PYG{+w}{ }\PYG{l+lScalar+lScalarPlain}{Miles}\PYG{l+lScalar+lScalarPlain}{ }\PYG{l+lScalar+lScalarPlain}{Mysterson}
\PYG{+w}{    }\PYG{n+nt}{affiliation}\PYG{p}{:}\PYG{+w}{ }\PYG{l+lScalar+lScalarPlain}{ubc}
\end{sphinxVerbatim}

\begin{sphinxVerbatim}[commandchars=\\\{\}]
\PYG{c+c1}{\PYGZsh{} Normalized}
\PYG{n+nt}{authors}\PYG{p}{:}
\PYG{+w}{  }\PYG{p+pIndicator}{\PYGZhy{}}\PYG{+w}{ }\PYG{n+nt}{name}\PYG{p}{:}\PYG{+w}{ }\PYG{l+lScalar+lScalarPlain}{Marissa}\PYG{l+lScalar+lScalarPlain}{ }\PYG{l+lScalar+lScalarPlain}{Myst}
\PYG{+w}{    }\PYG{n+nt}{affiliations}\PYG{p}{:}\PYG{+w}{ }\PYG{p+pIndicator}{[}\PYG{l+s}{\PYGZsq{}}\PYG{l+s}{ubc}\PYG{l+s}{\PYGZsq{}}\PYG{p+pIndicator}{,}\PYG{+w}{ }\PYG{l+s}{\PYGZsq{}}\PYG{l+s}{ACME}\PYG{n+nv}{ }\PYG{l+s}{Inc}\PYG{l+s}{\PYGZsq{}}\PYG{p+pIndicator}{]}
\PYG{+w}{  }\PYG{p+pIndicator}{\PYGZhy{}}\PYG{+w}{ }\PYG{n+nt}{name}\PYG{p}{:}\PYG{+w}{ }\PYG{l+lScalar+lScalarPlain}{Miles}\PYG{l+lScalar+lScalarPlain}{ }\PYG{l+lScalar+lScalarPlain}{Mysterson}
\PYG{+w}{    }\PYG{n+nt}{affiliations}\PYG{p}{:}\PYG{+w}{ }\PYG{p+pIndicator}{[}\PYG{l+s}{\PYGZsq{}}\PYG{l+s}{ubc}\PYG{l+s}{\PYGZsq{}}\PYG{p+pIndicator}{]}
\PYG{n+nt}{affiliations}\PYG{p}{:}
\PYG{+w}{  }\PYG{p+pIndicator}{\PYGZhy{}}\PYG{+w}{ }\PYG{n+nt}{id}\PYG{p}{:}\PYG{+w}{ }\PYG{l+lScalar+lScalarPlain}{ubc}
\PYG{+w}{    }\PYG{n+nt}{institution}\PYG{p}{:}\PYG{+w}{ }\PYG{l+lScalar+lScalarPlain}{University}\PYG{l+lScalar+lScalarPlain}{ }\PYG{l+lScalar+lScalarPlain}{of}\PYG{l+lScalar+lScalarPlain}{ }\PYG{l+lScalar+lScalarPlain}{British}\PYG{l+lScalar+lScalarPlain}{ }\PYG{l+lScalar+lScalarPlain}{Columbia}
\PYG{+w}{    }\PYG{n+nt}{ror}\PYG{p}{:}\PYG{+w}{ }\PYG{l+lScalar+lScalarPlain}{https://ror.org/03rmrcq20}
\PYG{+w}{    }\PYG{n+nt}{department}\PYG{p}{:}\PYG{+w}{ }\PYG{l+lScalar+lScalarPlain}{Earth,}\PYG{l+lScalar+lScalarPlain}{ }\PYG{l+lScalar+lScalarPlain}{Ocean}\PYG{l+lScalar+lScalarPlain}{ }\PYG{l+lScalar+lScalarPlain}{and}\PYG{l+lScalar+lScalarPlain}{ }\PYG{l+lScalar+lScalarPlain}{Atmospheric}\PYG{l+lScalar+lScalarPlain}{ }\PYG{l+lScalar+lScalarPlain}{Sciences}
\PYG{+w}{  }\PYG{p+pIndicator}{\PYGZhy{}}\PYG{+w}{ }\PYG{n+nt}{id}\PYG{p}{:}\PYG{+w}{ }\PYG{l+lScalar+lScalarPlain}{ACME}\PYG{l+lScalar+lScalarPlain}{ }\PYG{l+lScalar+lScalarPlain}{Inc}
\PYG{+w}{    }\PYG{n+nt}{name}\PYG{p}{:}\PYG{+w}{ }\PYG{l+lScalar+lScalarPlain}{ACME}\PYG{l+lScalar+lScalarPlain}{ }\PYG{l+lScalar+lScalarPlain}{Inc}
\end{sphinxVerbatim}
\begin{itemize}
\item {} 
\sphinxAtStartPar
\sphinxcode{\sphinxupquote{id}} : a string \sphinxhyphen{} local identifier that can be used to easily reference a repeated affiliation.

\item {} 
\sphinxAtStartPar
\sphinxcode{\sphinxupquote{name}} : a string \sphinxhyphen{} the affiliation name. Either \sphinxcode{\sphinxupquote{name}} or \sphinxcode{\sphinxupquote{institution}} is required.

\item {} 
\sphinxAtStartPar
\sphinxcode{\sphinxupquote{institution}} : a string \sphinxhyphen{} Name of an institution or organization (for example, a university or corporation). If your research group has a name, you can use both \sphinxcode{\sphinxupquote{name}} and \sphinxcode{\sphinxupquote{institution}}, however, at least one of these is required.

\item {} 
\sphinxAtStartPar
\sphinxcode{\sphinxupquote{department}} : a string \sphinxhyphen{} the affiliation department (e.g., Chemistry).

\item {} 
\sphinxAtStartPar
\sphinxcode{\sphinxupquote{doi, ror, isni, ringgold}} : identifiers for the affiliation (DOI, ROR, ISNI, and Ringgold). We suggest using https://ror.org if possible to search for your institution.

\end{itemize}

\begin{sphinxVerbatim}[commandchars=\\\{\}]
\PYG{n+nt}{affiliations}\PYG{p}{:}
\PYG{+w}{  }\PYG{p+pIndicator}{\PYGZhy{}}\PYG{+w}{ }\PYG{n+nt}{name}\PYG{p}{:}\PYG{+w}{ }\PYG{l+lScalar+lScalarPlain}{Boston}\PYG{l+lScalar+lScalarPlain}{ }\PYG{l+lScalar+lScalarPlain}{University}
\PYG{+w}{    }\PYG{n+nt}{ringgold}\PYG{p}{:}\PYG{+w}{ }\PYG{l+lScalar+lScalarPlain}{1846}
\PYG{+w}{    }\PYG{n+nt}{isni}\PYG{p}{:}\PYG{+w}{ }\PYG{l+lScalar+lScalarPlain}{0000}\PYG{l+lScalar+lScalarPlain}{ }\PYG{l+lScalar+lScalarPlain}{0004}\PYG{l+lScalar+lScalarPlain}{ }\PYG{l+lScalar+lScalarPlain}{1936}\PYG{l+lScalar+lScalarPlain}{ }\PYG{l+lScalar+lScalarPlain}{7558}
\PYG{+w}{    }\PYG{n+nt}{ror}\PYG{p}{:}\PYG{+w}{ }\PYG{l+lScalar+lScalarPlain}{05qwgg493}
\PYG{+w}{    }\PYG{n+nt}{doi}\PYG{p}{:}\PYG{+w}{ }\PYG{l+lScalar+lScalarPlain}{10.13039/100018578}
\end{sphinxVerbatim}
\begin{itemize}
\item {} 
\sphinxAtStartPar
\sphinxcode{\sphinxupquote{email}} : a string \sphinxhyphen{} email of the affiliation, required if \sphinxcode{\sphinxupquote{corresponding}} is \sphinxcode{\sphinxupquote{true}}.

\item {} 
\sphinxAtStartPar
\sphinxcode{\sphinxupquote{address, address}}, \sphinxcode{\sphinxupquote{city}}, \sphinxcode{\sphinxupquote{state}}, \sphinxcode{\sphinxupquote{postal code}}, and \sphinxcode{\sphinxupquote{country}} : affiliation address information, in place of \sphinxcode{\sphinxupquote{state}} you can use \sphinxcode{\sphinxupquote{province}} or \sphinxcode{\sphinxupquote{region}}.

\item {} 
\sphinxAtStartPar
\sphinxcode{\sphinxupquote{url}} : a string \sphinxhyphen{} website or homepage of the affiliation (\sphinxcode{\sphinxupquote{website}} is an alias).

\item {} 
\sphinxAtStartPar
\sphinxcode{\sphinxupquote{phone}} : a phone number, e.g., (301) 754 5766.

\item {} 
\sphinxAtStartPar
\sphinxcode{\sphinxupquote{fax}} : A fax number for the affiliation.

\item {} 
\sphinxAtStartPar
\sphinxcode{\sphinxupquote{collaboration}} : a boolean \sphinxhyphen{} indicate that the affiliation is a collaboration, for example, “MyST Contributors” can be both an affiliation and a listed author. This is used in certain templates as well as in \sphinxstylestrong{JATS}.

\end{itemize}


\section{Date}
\label{\detokenize{notebooks/02-myst.integration:date}}
\sphinxAtStartPar
The date field is a string and should conform to a valid Javascript data format. Examples of acceptable date formats are:

\begin{sphinxVerbatim}[commandchars=\\\{\}]
\PYG{l+lScalar+lScalarPlain}{2021\PYGZhy{}12\PYGZhy{}14T10:43:51.777Z}\PYG{l+lScalar+lScalarPlain}{ }\PYG{l+lScalar+lScalarPlain}{\PYGZhy{}}\PYG{l+lScalar+lScalarPlain}{ }\PYG{l+lScalar+lScalarPlain}{an}\PYG{l+lScalar+lScalarPlain}{ }\PYG{l+lScalar+lScalarPlain}{ISO}\PYG{l+lScalar+lScalarPlain}{ }\PYG{l+lScalar+lScalarPlain}{8601}\PYG{l+lScalar+lScalarPlain}{ }\PYG{l+lScalar+lScalarPlain}{calendar}\PYG{l+lScalar+lScalarPlain}{ }\PYG{l+lScalar+lScalarPlain}{date}\PYG{l+lScalar+lScalarPlain}{ }\PYG{l+lScalar+lScalarPlain}{extended}\PYG{l+lScalar+lScalarPlain}{ }\PYG{l+lScalar+lScalarPlain}{format,}\PYG{l+lScalar+lScalarPlain}{ }\PYG{l+lScalar+lScalarPlain}{or}
\PYG{l+lScalar+lScalarPlain}{14}\PYG{l+lScalar+lScalarPlain}{ }\PYG{l+lScalar+lScalarPlain}{Dec}\PYG{l+lScalar+lScalarPlain}{ }\PYG{l+lScalar+lScalarPlain}{2021}
\PYG{l+lScalar+lScalarPlain}{14}\PYG{l+lScalar+lScalarPlain}{ }\PYG{l+lScalar+lScalarPlain}{December}\PYG{l+lScalar+lScalarPlain}{ }\PYG{l+lScalar+lScalarPlain}{2021}
\PYG{l+lScalar+lScalarPlain}{2021,}\PYG{l+lScalar+lScalarPlain}{ }\PYG{l+lScalar+lScalarPlain}{December}\PYG{l+lScalar+lScalarPlain}{ }\PYG{l+lScalar+lScalarPlain}{14}
\PYG{l+lScalar+lScalarPlain}{2021}\PYG{l+lScalar+lScalarPlain}{ }\PYG{l+lScalar+lScalarPlain}{December}\PYG{l+lScalar+lScalarPlain}{ }\PYG{l+lScalar+lScalarPlain}{14}
\PYG{l+lScalar+lScalarPlain}{12/14/2021}\PYG{l+lScalar+lScalarPlain}{ }\PYG{l+lScalar+lScalarPlain}{\PYGZhy{}}\PYG{l+lScalar+lScalarPlain}{ }\PYG{l+lScalar+lScalarPlain}{MM/DD/YYYY}
\PYG{l+lScalar+lScalarPlain}{12\PYGZhy{}14\PYGZhy{}2021}\PYG{l+lScalar+lScalarPlain}{ }\PYG{l+lScalar+lScalarPlain}{\PYGZhy{}}\PYG{l+lScalar+lScalarPlain}{ }\PYG{l+lScalar+lScalarPlain}{MM\PYGZhy{}DD\PYGZhy{}YYYY}
\PYG{l+lScalar+lScalarPlain}{2022/12/14}\PYG{l+lScalar+lScalarPlain}{ }\PYG{l+lScalar+lScalarPlain}{\PYGZhy{}}\PYG{l+lScalar+lScalarPlain}{ }\PYG{l+lScalar+lScalarPlain}{YYYY/MM/DD}
\PYG{l+lScalar+lScalarPlain}{2022\PYGZhy{}12\PYGZhy{}14}\PYG{l+lScalar+lScalarPlain}{ }\PYG{l+lScalar+lScalarPlain}{\PYGZhy{}}\PYG{l+lScalar+lScalarPlain}{ }\PYG{l+lScalar+lScalarPlain}{YYYY\PYGZhy{}MM\PYGZhy{}DD}
\end{sphinxVerbatim}

\sphinxAtStartPar
Where the latter example in that list are valid \sphinxstylestrong{IETF\sphinxhyphen{}timestamps}.


\section{Licenses}
\label{\detokenize{notebooks/02-myst.integration:licenses}}
\sphinxAtStartPar
This field can be set to a string value directly or to a License object.

\sphinxAtStartPar
Available fields in the License object are \sphinxcode{\sphinxupquote{content}} and \sphinxcode{\sphinxupquote{code}} allowing licenses to be set separately for these two forms of content, as often different subsets of licenses are applicable to each. If you only wish to apply a single license to your page or project use the string form rather than an object.

\sphinxAtStartPar
String values for licenses should be a valid “Identifier” string from the \sphinxhref{https://spdx.org/licenses/}{SPDX License List}. Identifiers for well\sphinxhyphen{}known licenses are easily recognizable (e.g., \sphinxcode{\sphinxupquote{MIT}} or \sphinxcode{\sphinxupquote{BSD}}) and MyST will attempt to infer the specific identifier if an ambiguous license is specified (e.g., \sphinxcode{\sphinxupquote{GPL}} will be interpreted as \sphinxcode{\sphinxupquote{GPL\sphinxhyphen{}3.0+}} and a warning raised letting you know of this interpretation). Some common licenses are:

\begin{sphinxVerbatim}[commandchars=\\\{\}]
\PYG{c+c1}{\PYGZsh{} Common Content Licenses}
\PYG{l+lScalar+lScalarPlain}{CC\PYGZhy{}BY\PYGZhy{}4.0}
\PYG{l+lScalar+lScalarPlain}{CC\PYGZhy{}BY\PYGZhy{}SA\PYGZhy{}4.0}
\PYG{l+lScalar+lScalarPlain}{CC\PYGZhy{}BY\PYGZhy{}N\PYGZhy{}SA\PYGZhy{}4.0}
\PYG{l+lScalar+lScalarPlain}{CC0\PYGZhy{}1.0}
\end{sphinxVerbatim}

\begin{sphinxVerbatim}[commandchars=\\\{\}]
\PYG{c+c1}{\PYGZsh{} Common Code Licenses}
\PYG{l+lScalar+lScalarPlain}{MIT}
\PYG{l+lScalar+lScalarPlain}{BSD}
\PYG{l+lScalar+lScalarPlain}{GPL\PYGZhy{}3.0+}
\PYG{l+lScalar+lScalarPlain}{Apache\PYGZhy{}2.0}
\PYG{l+lScalar+lScalarPlain}{LGPL\PYGZhy{}3.0\PYGZhy{}or\PYGZhy{}later}
\PYG{l+lScalar+lScalarPlain}{AGPL}
\end{sphinxVerbatim}

\sphinxAtStartPar
By using the correct SPDX Identifier, your website will automatically use the appropriate icon for the license and link to the license definition.


\section{Funding}
\label{\detokenize{notebooks/02-myst.integration:funding}}
\sphinxAtStartPar
Funding frontmatter is able to contain multiple funding and open access statements, as well as award info.

\sphinxAtStartPar
It may be as simple as a single funding statement:

\begin{sphinxVerbatim}[commandchars=\\\{\}]
\PYG{n+nt}{funding}\PYG{p}{:}\PYG{+w}{ }\PYG{l+lScalar+lScalarPlain}{This}\PYG{l+lScalar+lScalarPlain}{ }\PYG{l+lScalar+lScalarPlain}{work}\PYG{l+lScalar+lScalarPlain}{ }\PYG{l+lScalar+lScalarPlain}{was}\PYG{l+lScalar+lScalarPlain}{ }\PYG{l+lScalar+lScalarPlain}{supported}\PYG{l+lScalar+lScalarPlain}{ }\PYG{l+lScalar+lScalarPlain}{by}\PYG{l+lScalar+lScalarPlain}{ }\PYG{l+lScalar+lScalarPlain}{University.}
\end{sphinxVerbatim}

\sphinxAtStartPar
Funding may also specify award id, name, sources (\sphinxcode{\sphinxupquote{affiliation object}} or \sphinxcode{\sphinxupquote{reference}}), investigators (\sphinxcode{\sphinxupquote{contributor objects}} or \sphinxcode{\sphinxupquote{references}}), and recipients (\sphinxcode{\sphinxupquote{contributor objects}} or \sphinxcode{\sphinxupquote{references}}).

\begin{sphinxVerbatim}[commandchars=\\\{\}]
\PYG{n+nt}{authors}\PYG{p}{:}
\PYG{+w}{  }\PYG{p+pIndicator}{\PYGZhy{}}\PYG{+w}{ }\PYG{n+nt}{id}\PYG{p}{:}\PYG{+w}{ }\PYG{l+lScalar+lScalarPlain}{auth0}
\PYG{+w}{    }\PYG{n+nt}{name}\PYG{p}{:}\PYG{+w}{ }\PYG{l+lScalar+lScalarPlain}{Jane}\PYG{l+lScalar+lScalarPlain}{ }\PYG{l+lScalar+lScalarPlain}{Doe}
\PYG{n+nt}{funding}\PYG{p}{:}
\PYG{+w}{  }\PYG{n+nt}{statement}\PYG{p}{:}\PYG{+w}{ }\PYG{l+lScalar+lScalarPlain}{This}\PYG{l+lScalar+lScalarPlain}{ }\PYG{l+lScalar+lScalarPlain}{work}\PYG{l+lScalar+lScalarPlain}{ }\PYG{l+lScalar+lScalarPlain}{was}\PYG{l+lScalar+lScalarPlain}{ }\PYG{l+lScalar+lScalarPlain}{supported}\PYG{l+lScalar+lScalarPlain}{ }\PYG{l+lScalar+lScalarPlain}{by}\PYG{l+lScalar+lScalarPlain}{ }\PYG{l+lScalar+lScalarPlain}{University.}
\PYG{+w}{  }\PYG{n+nt}{id}\PYG{p}{:}\PYG{+w}{ }\PYG{l+lScalar+lScalarPlain}{award\PYGZhy{}id\PYGZhy{}000}
\PYG{+w}{  }\PYG{n+nt}{name}\PYG{p}{:}\PYG{+w}{ }\PYG{l+lScalar+lScalarPlain}{My}\PYG{l+lScalar+lScalarPlain}{ }\PYG{l+lScalar+lScalarPlain}{Award}
\PYG{+w}{  }\PYG{n+nt}{sources}\PYG{p}{:}
\PYG{+w}{    }\PYG{p+pIndicator}{\PYGZhy{}}\PYG{+w}{ }\PYG{n+nt}{name}\PYG{p}{:}\PYG{+w}{ }\PYG{l+lScalar+lScalarPlain}{University}
\PYG{+w}{  }\PYG{n+nt}{investigators}\PYG{p}{:}
\PYG{+w}{    }\PYG{p+pIndicator}{\PYGZhy{}}\PYG{+w}{ }\PYG{n+nt}{name}\PYG{p}{:}\PYG{+w}{ }\PYG{l+lScalar+lScalarPlain}{John}\PYG{l+lScalar+lScalarPlain}{ }\PYG{l+lScalar+lScalarPlain}{Doe}
\PYG{+w}{  }\PYG{n+nt}{recipients}\PYG{p}{:}
\PYG{+w}{    }\PYG{p+pIndicator}{\PYGZhy{}}\PYG{+w}{ }\PYG{l+lScalar+lScalarPlain}{auth0}
\end{sphinxVerbatim}

\sphinxAtStartPar
Multiple funding objects with multiple awards are also possible:

\begin{sphinxVerbatim}[commandchars=\\\{\}]
\PYG{n+nt}{authors}\PYG{p}{:}
\PYG{+w}{  }\PYG{p+pIndicator}{\PYGZhy{}}\PYG{+w}{ }\PYG{n+nt}{id}\PYG{p}{:}\PYG{+w}{ }\PYG{l+lScalar+lScalarPlain}{auth0}
\PYG{+w}{    }\PYG{n+nt}{name}\PYG{p}{:}\PYG{+w}{ }\PYG{l+lScalar+lScalarPlain}{Jane}\PYG{l+lScalar+lScalarPlain}{ }\PYG{l+lScalar+lScalarPlain}{Doe}
\PYG{n+nt}{funding}\PYG{p}{:}
\PYG{+w}{  }\PYG{p+pIndicator}{\PYGZhy{}}\PYG{+w}{ }\PYG{n+nt}{statement}\PYG{p}{:}\PYG{+w}{ }\PYG{l+lScalar+lScalarPlain}{This}\PYG{l+lScalar+lScalarPlain}{ }\PYG{l+lScalar+lScalarPlain}{work}\PYG{l+lScalar+lScalarPlain}{ }\PYG{l+lScalar+lScalarPlain}{was}\PYG{l+lScalar+lScalarPlain}{ }\PYG{l+lScalar+lScalarPlain}{supported}\PYG{l+lScalar+lScalarPlain}{ }\PYG{l+lScalar+lScalarPlain}{by}\PYG{l+lScalar+lScalarPlain}{ }\PYG{l+lScalar+lScalarPlain}{University.}
\PYG{+w}{    }\PYG{n+nt}{awards}\PYG{p}{:}
\PYG{+w}{      }\PYG{p+pIndicator}{\PYGZhy{}}\PYG{+w}{ }\PYG{n+nt}{id}\PYG{p}{:}\PYG{+w}{ }\PYG{l+lScalar+lScalarPlain}{award\PYGZhy{}id\PYGZhy{}000}
\PYG{+w}{        }\PYG{n+nt}{name}\PYG{p}{:}\PYG{+w}{ }\PYG{l+lScalar+lScalarPlain}{My}\PYG{l+lScalar+lScalarPlain}{ }\PYG{l+lScalar+lScalarPlain}{First}\PYG{l+lScalar+lScalarPlain}{ }\PYG{l+lScalar+lScalarPlain}{Award}
\PYG{+w}{        }\PYG{n+nt}{sources}\PYG{p}{:}
\PYG{+w}{          }\PYG{p+pIndicator}{\PYGZhy{}}\PYG{+w}{ }\PYG{n+nt}{name}\PYG{p}{:}\PYG{+w}{ }\PYG{l+lScalar+lScalarPlain}{University}
\PYG{+w}{        }\PYG{n+nt}{investigators}\PYG{p}{:}
\PYG{+w}{          }\PYG{p+pIndicator}{\PYGZhy{}}\PYG{+w}{ }\PYG{n+nt}{name}\PYG{p}{:}\PYG{+w}{ }\PYG{l+lScalar+lScalarPlain}{John}\PYG{l+lScalar+lScalarPlain}{ }\PYG{l+lScalar+lScalarPlain}{Doe}
\PYG{+w}{        }\PYG{n+nt}{recipients}\PYG{p}{:}
\PYG{+w}{          }\PYG{p+pIndicator}{\PYGZhy{}}\PYG{+w}{ }\PYG{l+lScalar+lScalarPlain}{auth0}
\PYG{+w}{      }\PYG{p+pIndicator}{\PYGZhy{}}\PYG{+w}{ }\PYG{n+nt}{id}\PYG{p}{:}\PYG{+w}{ }\PYG{l+lScalar+lScalarPlain}{award\PYGZhy{}id\PYGZhy{}001}
\PYG{+w}{        }\PYG{n+nt}{name}\PYG{p}{:}\PYG{+w}{ }\PYG{l+lScalar+lScalarPlain}{My}\PYG{l+lScalar+lScalarPlain}{ }\PYG{l+lScalar+lScalarPlain}{Second}\PYG{l+lScalar+lScalarPlain}{ }\PYG{l+lScalar+lScalarPlain}{Award}
\PYG{+w}{        }\PYG{n+nt}{sources}\PYG{p}{:}
\PYG{+w}{          }\PYG{p+pIndicator}{\PYGZhy{}}\PYG{+w}{ }\PYG{n+nt}{name}\PYG{p}{:}\PYG{+w}{ }\PYG{l+lScalar+lScalarPlain}{University}
\PYG{+w}{        }\PYG{n+nt}{investigators}\PYG{p}{:}
\PYG{+w}{          }\PYG{p+pIndicator}{\PYGZhy{}}\PYG{+w}{ }\PYG{n+nt}{name}\PYG{p}{:}\PYG{+w}{ }\PYG{l+lScalar+lScalarPlain}{John}\PYG{l+lScalar+lScalarPlain}{ }\PYG{l+lScalar+lScalarPlain}{Doe}
\PYG{+w}{        }\PYG{n+nt}{recipients}\PYG{p}{:}
\PYG{+w}{          }\PYG{p+pIndicator}{\PYGZhy{}}\PYG{+w}{ }\PYG{l+lScalar+lScalarPlain}{auth0}
\PYG{+w}{  }\PYG{p+pIndicator}{\PYGZhy{}}\PYG{+w}{ }\PYG{n+nt}{statement}\PYG{p}{:}\PYG{+w}{ }\PYG{l+lScalar+lScalarPlain}{Open}\PYG{l+lScalar+lScalarPlain}{ }\PYG{l+lScalar+lScalarPlain}{access}\PYG{l+lScalar+lScalarPlain}{ }\PYG{l+lScalar+lScalarPlain}{was}\PYG{l+lScalar+lScalarPlain}{ }\PYG{l+lScalar+lScalarPlain}{supported}\PYG{l+lScalar+lScalarPlain}{ }\PYG{l+lScalar+lScalarPlain}{by}\PYG{l+lScalar+lScalarPlain}{ }\PYG{l+lScalar+lScalarPlain}{Consortium.}
\PYG{+w}{    }\PYG{n+nt}{open\PYGZus{}access}\PYG{p}{:}\PYG{+w}{ }\PYG{l+lScalar+lScalarPlain}{Users}\PYG{l+lScalar+lScalarPlain}{ }\PYG{l+lScalar+lScalarPlain}{are}\PYG{l+lScalar+lScalarPlain}{ }\PYG{l+lScalar+lScalarPlain}{allowed}\PYG{l+lScalar+lScalarPlain}{ }\PYG{l+lScalar+lScalarPlain}{to}\PYG{l+lScalar+lScalarPlain}{ }\PYG{l+lScalar+lScalarPlain}{reproduce}\PYG{l+lScalar+lScalarPlain}{ }\PYG{l+lScalar+lScalarPlain}{without}\PYG{l+lScalar+lScalarPlain}{ }\PYG{l+lScalar+lScalarPlain}{prior}\PYG{l+lScalar+lScalarPlain}{ }\PYG{l+lScalar+lScalarPlain}{permission}
\PYG{+w}{    }\PYG{n+nt}{awards}\PYG{p}{:}
\PYG{+w}{      }\PYG{p+pIndicator}{\PYGZhy{}}\PYG{+w}{ }\PYG{n+nt}{id}\PYG{p}{:}\PYG{+w}{ }\PYG{l+lScalar+lScalarPlain}{open\PYGZhy{}award\PYGZhy{}999}
\PYG{+w}{        }\PYG{n+nt}{sources}\PYG{p}{:}
\PYG{+w}{          }\PYG{p+pIndicator}{\PYGZhy{}}\PYG{+w}{ }\PYG{n+nt}{name}\PYG{p}{:}\PYG{+w}{ }\PYG{l+lScalar+lScalarPlain}{Consortium}
\end{sphinxVerbatim}


\section{Venue}
\label{\detokenize{notebooks/02-myst.integration:venue}}
\sphinxAtStartPar
The term \sphinxcode{\sphinxupquote{venue}} is borrowed from the \sphinxstylestrong{OpenAlex} API definition:

\sphinxAtStartPar
\sphinxstylestrong{Venues are where works are hosted}.

\sphinxAtStartPar
Available fields in the \sphinxcode{\sphinxupquote{venue}} object are \sphinxcode{\sphinxupquote{title}} and \sphinxcode{\sphinxupquote{url}}.

\sphinxAtStartPar
Some typical \sphinxcode{\sphinxupquote{venue}} values may be:

\begin{sphinxVerbatim}[commandchars=\\\{\}]
\PYG{n+nt}{venue}\PYG{p}{:}
\PYG{+w}{  }\PYG{n+nt}{title}\PYG{p}{:}\PYG{+w}{ }\PYG{l+lScalar+lScalarPlain}{Journal}\PYG{l+lScalar+lScalarPlain}{ }\PYG{l+lScalar+lScalarPlain}{of}\PYG{l+lScalar+lScalarPlain}{ }\PYG{l+lScalar+lScalarPlain}{Geophysics}
\PYG{+w}{  }\PYG{n+nt}{url}\PYG{p}{:}\PYG{+w}{ }\PYG{l+lScalar+lScalarPlain}{https://journal.geophysicsjournal.com}
\end{sphinxVerbatim}

\begin{sphinxVerbatim}[commandchars=\\\{\}]
\PYG{n+nt}{venue}\PYG{p}{:}
\PYG{+w}{  }\PYG{n+nt}{title}\PYG{p}{:}\PYG{+w}{ }\PYG{l+lScalar+lScalarPlain}{EuroSciPy}\PYG{l+lScalar+lScalarPlain}{ }\PYG{l+lScalar+lScalarPlain}{2022}
\PYG{+w}{  }\PYG{n+nt}{url}\PYG{p}{:}\PYG{+w}{ }\PYG{l+lScalar+lScalarPlain}{https://www.euroscipy.org/2022}
\end{sphinxVerbatim}


\section{Biblio}
\label{\detokenize{notebooks/02-myst.integration:biblio}}
\sphinxAtStartPar
The term \sphinxcode{\sphinxupquote{biblio}} is borrowed from the \sphinxstylestrong{OpenAlex} API definition:

\sphinxAtStartPar
\sphinxstylestrong{Old\sphinxhyphen{}timey bibliographic info for this work. This is mostly useful only in citation/reference contexts. These are all strings because sometimes you’ll get fun values like “Spring” and “Inside cover.”}

\sphinxAtStartPar
Available fields in the \sphinxcode{\sphinxupquote{biblio}} object are \sphinxcode{\sphinxupquote{volumne}}, \sphinxcode{\sphinxupquote{issue}}, \sphinxcode{\sphinxupquote{first\_page}} and \sphinxcode{\sphinxupquote{last\_page}}.

\sphinxAtStartPar
Some example \sphinxcode{\sphinxupquote{biblio}} values may be:

\begin{sphinxVerbatim}[commandchars=\\\{\}]
\PYG{n+nt}{biblio}\PYG{p}{:}
\PYG{+w}{  }\PYG{n+nt}{volume}\PYG{p}{:}\PYG{+w}{ }\PYG{l+s}{\PYGZsq{}}\PYG{l+s}{42}\PYG{l+s}{\PYGZsq{}}
\PYG{+w}{  }\PYG{n+nt}{issue}\PYG{p}{:}\PYG{+w}{ }\PYG{l+s}{\PYGZsq{}}\PYG{l+s}{3}\PYG{l+s}{\PYGZsq{}}
\PYG{+w}{  }\PYG{n+nt}{first\PYGZus{}page}\PYG{p}{:}\PYG{+w}{ }\PYG{l+s}{\PYGZsq{}}\PYG{l+s}{1}\PYG{l+s}{\PYGZsq{}}\PYG{+w}{ }\PYG{c+c1}{\PYGZsh{} can be a number or string}
\PYG{+w}{  }\PYG{n+nt}{last\PYGZus{}page}\PYG{p}{:}\PYG{+w}{ }\PYG{l+s}{\PYGZsq{}}\PYG{l+s}{99}\PYG{l+s}{\PYGZsq{}}\PYG{+w}{ }\PYG{c+c1}{\PYGZsh{} can be a number or string}
\end{sphinxVerbatim}

\begin{sphinxVerbatim}[commandchars=\\\{\}]
\PYG{n+nt}{biblio}\PYG{p}{:}
\PYG{+w}{  }\PYG{n+nt}{volume}\PYG{p}{:}\PYG{+w}{ }\PYG{l+s}{\PYGZsq{}}\PYG{l+s}{2022}\PYG{l+s}{\PYGZsq{}}
\PYG{+w}{  }\PYG{n+nt}{issue}\PYG{p}{:}\PYG{+w}{ }\PYG{l+lScalar+lScalarPlain}{Winter}
\PYG{+w}{  }\PYG{n+nt}{first\PYGZus{}page}\PYG{p}{:}\PYG{+w}{ }\PYG{l+lScalar+lScalarPlain}{Inside}\PYG{l+lScalar+lScalarPlain}{ }\PYG{l+lScalar+lScalarPlain}{cover}\PYG{+w}{ }\PYG{c+c1}{\PYGZsh{} can be a number or string}
\end{sphinxVerbatim}


\chapter{Settings}
\label{\detokenize{notebooks/02-myst.integration:settings}}
\sphinxAtStartPar
The \sphinxcode{\sphinxupquote{settings}} field in the project or page frontmatter allows you to change how the parsing, transforms, plugins, or other behaviors of mystmd.


\section{Available settings fields}
\label{\detokenize{notebooks/02-myst.integration:available-settings-fields}}\begin{itemize}
\item {} 
\sphinxAtStartPar
\sphinxcode{\sphinxupquote{output\_stderr}} : Remove, warn, or error on \sphinxcode{\sphinxupquote{stderr}} outputs (e.g., \sphinxcode{\sphinxupquote{DeprecationWarnings}}, \sphinxcode{\sphinxupquote{RuntimeWarnings}}).
\begin{itemize}
\item {} 
\sphinxAtStartPar
\sphinxcode{\sphinxupquote{show}} : (default) : show all \sphinxcode{\sphinxupquote{stderr}} (unless a \sphinxcode{\sphinxupquote{remove\sphinxhyphen{}stderr}} tag is present on  the cell).

\item {} 
\sphinxAtStartPar
\sphinxcode{\sphinxupquote{remove}} : remove all \sphinxcode{\sphinxupquote{stderr}} outputs.

\item {} 
\sphinxAtStartPar
\sphinxcode{\sphinxupquote{remove\sphinxhyphen{}warn}} or \sphinxcode{\sphinxupquote{remove\sphinxhyphen{}error}} : remove all \sphinxcode{\sphinxupquote{stderr}}, and log a warning or error.

\item {} 
\sphinxAtStartPar
\sphinxcode{\sphinxupquote{warn}} or “error” : log a warning or error if a \sphinxcode{\sphinxupquote{stderr}} is found.

\end{itemize}

\item {} 
\sphinxAtStartPar
\sphinxcode{\sphinxupquote{output\_stdout}} : Remove, warn or error on \sphinxcode{\sphinxupquote{stdout}} outputs. (e.g., long text outputs, like text\sphinxhyphen{}based progress bars).
\begin{itemize}
\item {} 
\sphinxAtStartPar
\sphinxcode{\sphinxupquote{show}} : (default) : show all \sphinxcode{\sphinxupquote{stdout}} (unless a \sphinxcode{\sphinxupquote{remove\sphinxhyphen{}stdout}} tag is present on the cell).

\item {} 
\sphinxAtStartPar
\sphinxcode{\sphinxupquote{remove\sphinxhyphen{}warn}} or \sphinxcode{\sphinxupquote{remove\sphinxhyphen{}error}} : remove all \sphinxcode{\sphinxupquote{stdout}}, and log a warning or error.

\item {} 
\sphinxAtStartPar
\sphinxcode{\sphinxupquote{warn}} or “error” : log a warning or error if a \sphinxcode{\sphinxupquote{stdout}} is found.

\end{itemize}

\item {} 
\sphinxAtStartPar
\sphinxcode{\sphinxupquote{output\_matplotlib\_strings}} : Remove, warn, or error on matplotlib strings outputs. (e.g., \textless{}Figure size 720x576 with 1 Axes\textgreater{} or Text(0.5, 0.98, ‘Test 1’)). These can also be suppressed by ending your cell content with a semicolon in Jupyter Notebooks. The default is to remove these and warn (\sphinxcode{\sphinxupquote{remove\sphinxhyphen{}warn}}).
\begin{itemize}
\item {} 
\sphinxAtStartPar
\sphinxcode{\sphinxupquote{show}} : show al matplotlib strings in outputs.

\item {} 
\sphinxAtStartPar
\sphinxcode{\sphinxupquote{remove}} : remove all matplotlib strings in outputs

\item {} 
\sphinxAtStartPar
\sphinxcode{\sphinxupquote{remove\sphinxhyphen{}warn}} : (default) or \sphinxcode{\sphinxupquote{remove\sphinxhyphen{}error}} : remove all matplotlib strings in outputs, and log a warning or error.

\item {} 
\sphinxAtStartPar
\sphinxcode{\sphinxupquote{warn}} or “error” : log a warning or error if matplotlib strings in outputs.

\end{itemize}

\end{itemize}


\chapter{More From MyST Markdown}
\label{\detokenize{notebooks/02-myst.integration:more-from-myst-markdown}}
\sphinxAtStartPar
Most of this Notebook has been focused on the use of frontmatter to enrich the quality and dynamism of our projects. But from skimming through the extensive documentation, it is clear that MyST Markdown Tools might have more value than just allowing richer metadata in Jupyter Notebooks.

\sphinxAtStartPar
According to the \sphinxhref{https://mystmd.org/guide}{MyST Markdown homepage}, their tools are designed to revolutionize scientific communication in its various forms. Labeling itself an “authoring framework,” MyST products handle blogs, online books, scientific papers, reports, and journal articles.


\section{Key Features}
\label{\detokenize{notebooks/02-myst.integration:key-features}}\begin{itemize}
\item {} 
\sphinxAtStartPar
\sphinxstylestrong{Rabbit\sphinxhyphen{}hole links} allow you to get information to your reader as fast as possible and they can deep\sphinxhyphen{}dive all the way to computations, code, and interactive figures.

\item {} 
\sphinxAtStartPar
\sphinxstylestrong{Live graphs} can be embedded directly in your documentation or articles with computation backed by Jupyter \textendash{}running locally, on Binder, or directly in your browser.

\item {} 
\sphinxAtStartPar
\sphinxstylestrong{JupyterLab support} for MyST comes with inline computations, support for \sphinxcode{\sphinxupquote{ipywidgets}}, matplotlib sparklines, editable task\sphinxhyphen{}lists, rich frontmatter (as we know already), and beautiful typography and other elements like dropdowns, grids and cards.

\item {} 
\sphinxAtStartPar
\sphinxstylestrong{Export to PDF} is easy and they support hundreds of different journals out of the box, see \sphinxhref{https://github.com/myst-templates}{myst\sphinxhyphen{}templates}. You can also export to Microsoft Word or JATS, which is used in scientific publishing.

\end{itemize}


\section{Project Goals}
\label{\detokenize{notebooks/02-myst.integration:project-goals}}
\sphinxAtStartPar
MyST is part of the \sphinxhref{https://executablebooks.org/}{Executable Books} organization, which is a community driven project to improve scientific communication, including integrations into Jupyter Notebooks and computational results.


\subsection{Built for Science}
\label{\detokenize{notebooks/02-myst.integration:built-for-science}}
\sphinxAtStartPar
Extend Markdown with equations, cross\sphinxhyphen{}references, citations, and export to preprint or rich, interactive website or book.


\subsection{Dynamic Documents}
\label{\detokenize{notebooks/02-myst.integration:dynamic-documents}}
\sphinxAtStartPar
Make your pages interactive by connecting to custom JupyterHubs, public Binders, or even Python directly in your browser.


\subsection{Fast \& Accessible}
\label{\detokenize{notebooks/02-myst.integration:fast-accessible}}
\sphinxAtStartPar
Publish next\sphinxhyphen{}generation articles and books that are beautifully designed, without compromising on accessibility or performance.


\subsection{Technical Goals}
\label{\detokenize{notebooks/02-myst.integration:technical-goals}}\begin{itemize}
\item {} 
\sphinxAtStartPar
\sphinxcode{\sphinxupquote{mystmd}} is a Javascript parser and command line tool for working with MyST Markdown.

\item {} 
\sphinxAtStartPar
Parse MyST into a standardized \sphinxstylestrong{AST}, that follows the MyST Spec.

\item {} 
\sphinxAtStartPar
Translate and render MyST into:
\begin{itemize}
\item {} 
\sphinxAtStartPar
Modern \sphinxstylestrong{interactive websites}, using React.

\item {} 
\sphinxAtStartPar
PDFs and LaTeX documents, with specific templates for over 400 journals.

\item {} 
\sphinxAtStartPar
Microsoft Word \sphinxstylestrong{export}.

\end{itemize}

\item {} 
\sphinxAtStartPar
Provide functionality for \sphinxstyleemphasis{cross\sphinxhyphen{}referencing}, \sphinxstyleemphasis{external structured links}, and \sphinxstyleemphasis{scientific citations}.

\end{itemize}


\section{Installing the MyST Markdown CLI}
\label{\detokenize{notebooks/02-myst.integration:installing-the-myst-markdown-cli}}
\sphinxAtStartPar
\sphinxcode{\sphinxupquote{mystmd}} is a command line interface that provides modern tooling for technical writing, reproducible science, and creating scientific \& technical websites. To get started, install \sphinxcode{\sphinxupquote{mystmd}}.


\subsection{Prerequisites \sphinxhyphen{} Install Node}
\label{\detokenize{notebooks/02-myst.integration:prerequisites-install-node}}
\sphinxAtStartPar
You should have these programs installed already from our last Notebook:
\begin{itemize}
\item {} 
\sphinxAtStartPar
\sphinxcode{\sphinxupquote{Node.js}} version \textgreater{}=18.0.0

\item {} 
\sphinxAtStartPar
\sphinxcode{\sphinxupquote{npm}} version \textgreater{}=7.0.0

\item {} 
\sphinxAtStartPar
A code and notebook editor (\sphinxcode{\sphinxupquote{VS Code}} and \sphinxcode{\sphinxupquote{Jupyter Lab}} for notebooks).

\end{itemize}


\subsection{Install the MyST CLI}
\label{\detokenize{notebooks/02-myst.integration:install-the-myst-cli}}
\sphinxAtStartPar
Install node https://nodejs.org/ or through conda.

\begin{sphinxuseclass}{cell}
\begin{sphinxuseclass}{cell_input}
\begin{sphinxVerbatim}[commandchars=\\\{\}]
\PYG{n}{conda} \PYG{n}{install} \PYG{o}{\PYGZhy{}}\PYG{n}{c} \PYG{n}{conda}\PYG{o}{\PYGZhy{}}\PYG{n}{forge} \PYG{l+s+s1}{\PYGZsq{}}\PYG{l+s+s1}{nodejs\PYGZgt{}=20,\PYGZlt{}21}\PYG{l+s+s1}{\PYGZsq{}}
\end{sphinxVerbatim}

\end{sphinxuseclass}
\begin{sphinxuseclass}{cell_output}
\begin{sphinxVerbatim}[commandchars=\\\{\}]
zsh:1: 20, not found

Note: you may need to restart the kernel to use updated packages.
\end{sphinxVerbatim}

\end{sphinxuseclass}
\end{sphinxuseclass}
\sphinxAtStartPar
Then install \sphinxcode{\sphinxupquote{mystmd}}:

\begin{sphinxuseclass}{cell}
\begin{sphinxuseclass}{cell_input}
\begin{sphinxVerbatim}[commandchars=\\\{\}]
\PYG{n}{conda} \PYG{n}{install} \PYG{n}{mystmd} \PYG{o}{\PYGZhy{}}\PYG{n}{c} \PYG{n}{conda}\PYG{o}{\PYGZhy{}}\PYG{n}{forge}
\end{sphinxVerbatim}

\end{sphinxuseclass}
\begin{sphinxuseclass}{cell_output}
\begin{sphinxVerbatim}[commandchars=\\\{\}]
Channels:
 \PYGZhy{} conda\PYGZhy{}forge
 \PYGZhy{} defaults
Platform: osx\PYGZhy{}arm64
Collecting package metadata (repodata.json): done
Solving environment: done

\PYGZsh{} All requested packages already installed.


Note: you may need to restart the kernel to use updated packages.
\end{sphinxVerbatim}

\end{sphinxuseclass}
\end{sphinxuseclass}

\section{Circle Back at a Later Date}
\label{\detokenize{notebooks/02-myst.integration:circle-back-at-a-later-date}}
\sphinxAtStartPar
All of the promising features of MyST have certainly piqued my interests though much of the functionality is beyond the scope of the current state of my many projects. I will certainly be returning to this documentation when I begin publishing my Ancestry Research findings and as academic or professional opportunities come my way.

\sphinxAtStartPar
For now, we will focus our attention on producing thorough metadata through page\sphinxhyphen{}level directives and potentially project level\sphinxhyphen{}directives as we become more comfortable with the Jupyter Notebook system as a whole.

\sphinxAtStartPar
Let’s run back through this and the previous Notebook to input some quick metadata and move on to the next topic.

\sphinxstepscope


\chapter{Configuring Sphinx to Render HTML \& PDF Versions of Jupyter Notebooks}
\label{\detokenize{notebooks/03-sphinx-config:configuring-sphinx-to-render-html-pdf-versions-of-jupyter-notebooks}}\label{\detokenize{notebooks/03-sphinx-config::doc}}
\begin{sphinxadmonition}{note}{Objectives}

\sphinxAtStartPar
The objectives for this notebook are as follows:
\begin{itemize}
\item {} 
\sphinxAtStartPar
Integrate MyST Markdown and Sphinx documentation tools to produce a beautifully rendered HTML page on GitHub pages as well as a nicely formatted LaTeX\sphinxhyphen{}generated PDF.

\item {} 
\sphinxAtStartPar
Understand how to properly configure and add metadata to the document as a whole as well as at the notebook\sphinxhyphen{}level.

\item {} 
\sphinxAtStartPar
Begin to customize the entire pipeline so that it is both automated and efficient.

\end{itemize}
\end{sphinxadmonition}


\section{Prerequisites}
\label{\detokenize{notebooks/03-sphinx-config:prerequisites}}
\sphinxAtStartPar
It is highly recommended that you use the \sphinxcode{\sphinxupquote{conda}} environment scripts found in the \sphinxcode{\sphinxupquote{scripts/}} folder. They can be run from the root directory like this:

\begin{sphinxVerbatim}[commandchars=\\\{\}]
chmod\PYG{+w}{ }\PYGZhy{}R\PYG{+w}{ }\PYG{l+m}{755}\PYG{+w}{ }scripts/
./scripts/02\PYGZhy{}build\PYGZhy{}env.sh\PYG{+w}{ }\PYGZlt{}PATH\PYGZus{}TO\PYGZus{}ROOT\PYGZgt{}
\end{sphinxVerbatim}

\begin{sphinxadmonition}{note}{Note}

\sphinxAtStartPar
Note that the \sphinxcode{\sphinxupquote{\textless{}PATH\_TO\_ROOT\textgreater{}}} is passed here as an argument to the script, therefore it must be absolute. You can find this variable by simple typing \sphinxcode{\sphinxupquote{pwd}} from the root directory of the project.
\end{sphinxadmonition}

\sphinxAtStartPar
Close all terminals and start a fresh session. To be certain that the environment you just built is your active \sphinxcode{\sphinxupquote{conda}} environment:

\begin{sphinxVerbatim}[commandchars=\\\{\}]
conda\PYG{+w}{ }activate\PYG{+w}{ }jupyter
\end{sphinxVerbatim}

\sphinxAtStartPar
You can now navigate to your project root folder and issue the command to open with VS Code:

\begin{sphinxVerbatim}[commandchars=\\\{\}]
code\PYG{+w}{ }Jupyter\PYGZhy{}Notebooks\PYGZhy{}MyST\PYGZhy{}and\PYGZhy{}Sphinx/
\end{sphinxVerbatim}

\sphinxAtStartPar
A fresh instance of your project will open in VS Code. Now enter the command pallet \sphinxcode{\sphinxupquote{CMD\sphinxhyphen{}Shift\sphinxhyphen{}P}} and select \sphinxcode{\sphinxupquote{Python: Select Interpreter}}. Choose the version with the name of your \sphinxcode{\sphinxupquote{conda}} environment, it should be named \sphinxcode{\sphinxupquote{jupyter}}.

\sphinxAtStartPar
Next create a \sphinxcode{\sphinxupquote{docs/}} folder in the root directory and inside of it, create a \sphinxcode{\sphinxupquote{notebooks/}} folder. This will be where we keep all of our notebooks.

\sphinxAtStartPar
Start your first Jupyter notebook by creating a file with the extension \sphinxcode{\sphinxupquote{.ipynb}} and open that file. Create a new Markdown cell and type some dummy info. We will need content to test our configuration of Sphinx and the other tools.

\sphinxAtStartPar
Now we can begin the deep\sphinxhyphen{}dive into the Sphinx and MyST\sphinxhyphen{}NB configuration process.


\section{Getting Started with Sphinx}
\label{\detokenize{notebooks/03-sphinx-config:getting-started-with-sphinx}}
\begin{sphinxadmonition}{note}{Documentation}

\sphinxAtStartPar
For the full Sphinx Documentation visit\sphinxhref{https://www.sphinx-doc.org/en/master/index.html}{ sphinx\sphinxhyphen{}doc.org}. There are a lot of options… Sphinx is highly complex so I have tried to distill it down in these notebooks for you. Still, the documentation will provide all configuration options available for a more custom experience.
\end{sphinxadmonition}

\sphinxAtStartPar
I used a wonderful OBM Research Group tutorial %
\begin{footnote}[1]\sphinxAtStartFootnote
Tutorial: \sphinxhref{https://obm.physics.metu.edu.tr/node/65}{Sphinx environment with Conda}
%
\end{footnote} to configure most of Sphinx, though I tried my hand manually and had my fair share of trial and error.


\bigskip\hrule\bigskip


\sphinxstepscope

\begin{sphinxVerbatim}[commandchars=\\\{\}]
\PYG{n+nt}{title}\PYG{p}{:}\PYG{+w}{ }\PYG{l+lScalar+lScalarPlain}{Build}\PYG{l+lScalar+lScalarPlain}{ }\PYG{l+lScalar+lScalarPlain}{Custom}\PYG{l+lScalar+lScalarPlain}{ }\PYG{l+lScalar+lScalarPlain}{Kernel}
\PYG{n+nt}{description}\PYG{p}{:}\PYG{+w}{ }\PYG{l+lScalar+lScalarPlain}{This}\PYG{l+lScalar+lScalarPlain}{ }\PYG{l+lScalar+lScalarPlain}{Notebook}\PYG{l+lScalar+lScalarPlain}{ }\PYG{l+lScalar+lScalarPlain}{will}\PYG{l+lScalar+lScalarPlain}{ }\PYG{l+lScalar+lScalarPlain}{detail}\PYG{l+lScalar+lScalarPlain}{ }\PYG{l+lScalar+lScalarPlain}{how}\PYG{l+lScalar+lScalarPlain}{ }\PYG{l+lScalar+lScalarPlain}{to}\PYG{l+lScalar+lScalarPlain}{ }\PYG{l+lScalar+lScalarPlain}{implement}\PYG{l+lScalar+lScalarPlain}{ }\PYG{l+lScalar+lScalarPlain}{a}\PYG{l+lScalar+lScalarPlain}{ }\PYG{l+lScalar+lScalarPlain}{customer}\PYG{l+lScalar+lScalarPlain}{ }\PYG{l+lScalar+lScalarPlain}{kernel}\PYG{l+lScalar+lScalarPlain}{ }\PYG{l+lScalar+lScalarPlain}{with}\PYG{l+lScalar+lScalarPlain}{ }\PYG{l+lScalar+lScalarPlain}{support}\PYG{l+lScalar+lScalarPlain}{ }\PYG{l+lScalar+lScalarPlain}{for}\PYG{l+lScalar+lScalarPlain}{ }\PYG{l+lScalar+lScalarPlain}{IPython}\PYG{l+lScalar+lScalarPlain}{ }\PYG{l+lScalar+lScalarPlain}{which}\PYG{l+lScalar+lScalarPlain}{ }\PYG{l+lScalar+lScalarPlain}{Xeus\PYGZhy{}Lua}\PYG{l+lScalar+lScalarPlain}{ }\PYG{l+lScalar+lScalarPlain}{does}\PYG{l+lScalar+lScalarPlain}{ }\PYG{l+lScalar+lScalarPlain}{not}\PYG{l+lScalar+lScalarPlain}{ }\PYG{l+lScalar+lScalarPlain}{do}\PYG{l+lScalar+lScalarPlain}{ }\PYG{l+lScalar+lScalarPlain}{out\PYGZhy{}of\PYGZhy{}the\PYGZhy{}box.}
\PYG{n+nt}{date}\PYG{p}{:}\PYG{+w}{ }\PYG{l+lScalar+lScalarPlain}{2024\PYGZhy{}01\PYGZhy{}04}
\PYG{n+nt}{jupytext}\PYG{p}{:}
\PYG{+w}{    }\PYG{n+nt}{formats}\PYG{p}{:}\PYG{+w}{ }\PYG{l+lScalar+lScalarPlain}{md:myst}
\PYG{+w}{    }\PYG{l+lScalar+lScalarPlain}{text\PYGZhy{}representation}
\end{sphinxVerbatim}

\begin{sphinxuseclass}{cell}
\begin{sphinxuseclass}{cell_input}
\begin{sphinxVerbatim}[commandchars=\\\{\}]
\PYG{o}{\PYGZpc{}\PYGZpc{}html}
\PYG{p}{\PYGZlt{}}\PYG{n+nt}{style}\PYG{p}{\PYGZgt{}}
\PYG{+w}{    }\PYG{n+nt}{body}\PYG{+w}{ }\PYG{p}{\PYGZob{}}
\PYG{+w}{        }\PYG{n+nv}{\PYGZhy{}\PYGZhy{}vscode\PYGZhy{}font\PYGZhy{}family}\PYG{p}{:}\PYG{+w}{ }\PYG{l+s+s2}{\PYGZdq{}lmroman17\PYGZhy{}regular\PYGZdq{}}
\PYG{+w}{    }\PYG{p}{\PYGZcb{}}
\PYG{p}{\PYGZlt{}}\PYG{p}{/}\PYG{n+nt}{style}\PYG{p}{\PYGZgt{}}
\end{sphinxVerbatim}

\end{sphinxuseclass}
\begin{sphinxuseclass}{cell_output}
\begin{sphinxVerbatim}[commandchars=\\\{\}]
\PYGZlt{}IPython.core.display.HTML object\PYGZgt{}
\end{sphinxVerbatim}

\end{sphinxuseclass}
\end{sphinxuseclass}

\chapter{Building a Custom Jupyter Notebook Kernel}
\label{\detokenize{notebooks/04-build-custom-kernel:building-a-custom-jupyter-notebook-kernel}}\label{\detokenize{notebooks/04-build-custom-kernel::doc}}
\sphinxAtStartPar
I found when starting the \sphinxhref{https://www.udemy.com/share/10a9Qq3@bsv9B0\_-y2oIqI8dNeFs0eOv7PG5VL1JESM-yN0Mw9HtaVPeFRODH5bI3g\_TOZw=/}{Lua Programming Udemy course} that the stock \sphinxcode{\sphinxupquote{xeus\sphinxhyphen{}lua}} kernel was not dynamic enough to allow so\sphinxhyphen{}called \sphinxstyleemphasis{magic} commands offered by the IPython Kernel. Specifically, I could not run the \sphinxcode{\sphinxupquote{lua}} code as files that receive user input.

\sphinxAtStartPar
In reaction \textendash{}and at risk of becoming too granular\textendash{} I am taking this opportunity to learn how to build my own custom Jupyter kernel.

\sphinxAtStartPar
:::\{important\} Objectives
The objective of this exercise is to build a custom \sphinxcode{\sphinxupquote{lua}} kernel that allows for the use of magic \sphinxcode{\sphinxupquote{shell}} commands within the \sphinxcode{\sphinxupquote{lua}} code cells.
:::


\section{Using \sphinxstyleliteralintitle{\sphinxupquote{xeus}} to Author Custom Kernels}
\label{\detokenize{notebooks/04-build-custom-kernel:using-xeus-to-author-custom-kernels}}
\sphinxAtStartPar
\sphinxcode{\sphinxupquote{xeus}} enables custom kernel authors to implement Jupyter kernels more easily. It takes the burden of implementing the Jupyter Kernel protocol so developers can focus on implementing the interpreter part of the kernel.

\sphinxAtStartPar
\sphinxincludegraphics[width=1368\sphinxpxdimen,height=1236\sphinxpxdimen]{{protocol-diagram}.jpg}


\section{Implementing a Kernel}
\label{\detokenize{notebooks/04-build-custom-kernel:implementing-a-kernel}}
\sphinxAtStartPar
In most of the cases, the base kernel implementation is enough, and creating a kernel only means implementing the interpreter part.

\sphinxAtStartPar
The structure of our project should at least look like the following:

\sphinxAtStartPar
\sphinxincludegraphics[width=1250\sphinxpxdimen,height=822\sphinxpxdimen]{{planned-dir}.jpg}


\chapter{Indices and tables}
\label{\detokenize{index:indices-and-tables}}\begin{itemize}
\item {} 
\sphinxAtStartPar
\DUrole{xref,std,std-ref}{genindex}

\item {} 
\sphinxAtStartPar
\DUrole{xref,std,std-ref}{modindex}

\item {} 
\sphinxAtStartPar
\DUrole{xref,std,std-ref}{search}

\end{itemize}



\renewcommand{\indexname}{Index}
\printindex
\end{document}